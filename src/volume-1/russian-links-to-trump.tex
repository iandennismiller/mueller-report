\section{Russian Government Links to and Contacts with The Trump Campaign}

The Office identified multiple contacts - "links," in the words of the Appointment Order - between Trump Campaign officials and individuals with ties to the Russian government.
The Office investigated whether those contacts constituted a third avenue of attempted Russian interference with or influence on the 2016 presidential election.
In particular, the investigation examined whether these contacts involved or resulted in coordination or a conspiracy with the Trump Campaign and Russia, including with respect to Russia providing assistance to the Campaign in exchange for any sort of favorable treatment in the future.
Based on the available information, the investigation did not establish such coordination.

This Section describes the principal links between the Trump Campaign and individuals with ties to the Russian government, including some contacts with Campaign officials or associates that have been publicly reported to involve Russian contacts.
Each subsection begins with an overview of the Russian contact at issue and then describes in detail the relevant facts, which are generally presented in chronological order, beginning with the early months of the Campaign and extending through the post-election, transition period.

\subsection{Campaign Period (September 2015 - November 8, 2016)}

Russian-government-connected individuals and media entities began showing interest in Trump's campaign in the months after he announced his candidacy in June 2015.
Because Trump's status as a public figure at the time was attributable in large part to his prior business and entertainment dealings, this Office investigated whether a business contact with Russia-linked individuals and entities during the campaign period -- the Trump Tower Moscow project, see Volume I, Section IV.A.1, infra -- led to or involved coordination of election assistance.

Outreach from individuals with ties to Russia continued in the spring and summer of 2016, when Trump was moving toward -- and eventually becoming -- the Republican nominee for President.
As set forth below, the Office also evaluated a series of links during this period: outreach to two of Trump's then-recently named foreign policy advisors, including a representation that Russia had "dirt" on Clinton in the form of thousands of emails (VolumeI, Sections IV.A.2 \& IV.A.3); dealings with a D.C.-based think tank that specializes in Russia and has connections with its government (Volume I, Section IV.A.4); a meeting at Trump Tower between the Campaign and a Russian lawyer promising dirt on candidate Clinton that was "part of Russia and its government's support for [Trump]" (Volume I, Section IV.A.5); events at the Republican National Convention (Volume I, Section IV.A.6); post-Convention contacts between Trump Campaign officials and Russia's ambassador to the United States (Volume I, Section IV.A.7); and contacts through campaign chairman Paul Manafort, who had previously worked for a Russian oligarch and a pro-Russian political party in Ukraine (Volume I, Section IV.A.8).

\subsubsection{Trump Tower Moscow Project}

The Trump Organization has pursued and completed projects outside the United States as part of its real estate portfolio.
Some projects have involved the acquisition and ownership (through subsidiary corporate structures) of property.
In other cases, the Trump Organization has executed licensing deals with real estate developers and management companies, often local to the country where the project was located.

Between at least 2013 and 2016, the Trump Organization explored a  similar licensing deal in Russia involving the construction of a Trump-branded property in Moscow.
The project, commonly referred to as a "Trump Tower Moscow" or "Trump Moscow" project,  anticipated a combination of commercial, hotel, and residential properties all within the same building.
Between 2013 and June 2016, several employees of the Trump Organization, including then-president of the organization Donald J. Trump, pursued a  Moscow deal with several Russian counterparties.
From the fall of 2015 until the middle of 2016, Michael Cohen spearheaded the Trump Organization's pursuit of a  Trump Tower Moscow project, including by reporting on the project's status to candidate Trump and other executives in the Trump Organization.

\paragraph{Trump Tower Moscow Venture with the Crocus Group (2013-2014)}

The Trump Organization and the Crocus Group, a Russian real estate conglomerate owned and controlled by Aras Agalarov, began discussing a Russia-based real estate project shortly after the conclusion of the 2013 Miss Universe pageant in Moscow.
Donald J. Trump Jr. served as the primary negotiator on behalf of the Trump Organization; Emin Agalarov (son of Aras Agalarov) and Irakli "Ike" Kaveladze represented the Crocus Group during negotiations, with the occasional assistance of Robe1t Goldstone.

In December 2013, Kaveladze and Trump Jr. negotiated and signed preliminary terms of an agreement for the Trump Tower Moscow project.
On December 23, 2013, after discussions with Donald J. Trump, the Trump Organization agreed to accept an arrangement whereby the organization received a flat 3.5\% commission on all sales, with no licensing fees or incentives.
The parties negotiated a letter of intent during January and February 2014.

From January 2014 through November 2014, the Trump Organization and Crocus Group discussed development plans for the Moscow project.
Some time before January 24, 2014, the Crocus Group sent the Trump Organization a proposal for a 800-unit, 194-meter building to be constructed at an Agalarov-owned site in Moscow called "Crocus City," which had also been the site of the Miss Universe pageant.
In February 2014, Ivanka Trump met with Emin Agalarov and toured the Crocus City site during a visit to Moscow.
From March 2014 through July 2014, the groups discussed "design standards" and other architectural elements.
For example, in July 2014, members of the Trump Organization sent Crocus Group counterparties questions about the "demographics of these prospective buyers" in the Crocus City area, the development of neighboring parcels in Crocus City, and concepts for redesigning portions of the building.
In August 2014, the Trump Organization requested specifications for a competing Marriott-branded tower being built in Crocus City.

Beginning in September 2014, the Trump Organization stopped responding in a timely fashion to correspondence and proposals from the Crocus Group.
Communications between the two groups continued through November 2014 with decreasing frequency; what appears to be the last communication is dated November 24, 2014.
The project appears not to have developed past the planning stage, and no construction occurred.

\paragraph{Communications with I.C. Expert Investment Company and Giorgi Rtskhiladze (Summer and Fall 2015)}

In the late summer of 2015, the Trump Organization received a new inquiry about pursuing a Trump Tower project in Moscow.
In approximately September 2015, Felix Sater, a New York-based real estate advisor, contacted Michael Cohen, then-executive vice president of the Trump Organization and special counsel to Donald J. Trump.
Sater had previously worked with the Trump Organization and advised it on a number of domestic and international projects.
Sater had explored the possibility of a Trump Tower project in Moscow while working with the Trump Organiz ation and therefore knew of the organization's general interest in completing a  deal there.
Sater had also served as an informal agent of the Trump Organization in Moscow previously  and had accompanied lvanka Trump and Donald Trump Jr. to Moscow in the mid-2000s.

Sater contacted Cohen on behalf of I.C. Expert Investment Company (I.C. Expert), a Russian real-estate development corporation controlled by Andrei Vladimirovich Rozov.
Sater had known Rozov since approximately 2007 and, in 2014, had served as an agent on behalf of Rozov during Rozov's purchase of a building in New York City.
Sater later contacted Rozov and proposed that I.C. Expert pursue a Trump Tower Moscow project in which I.C. Expert would license the name and brand from the Trump Organization but construct the building on its own. Sater worked on the deal with Rozov and another.employee of I.C. Expert.

Cohen was the only Trump Organization representative to negotiate directly with I.C. Expert or its agents.
In approximately September 2015, Cohen obtained approval to negotiate with I.C. Expert from candidate Trump, who was then president of the Trump Organization.
Cohen provided updates directly to Trump about the project throughout 2015 and into 2016, assuring him the project was continuing.
Cohen also discussed the Trump Moscow project with Ivanka Trump as to design elements (such as possible architects to use for the project) and Donald J. Trump Jr. (about his experience in Moscow and possible involvement in the project) during the fall of 2015.

Also during the fall of 2015, Cohen communicated about the Trump Moscow proposal with Giorgi Rtskhiladze, a business executive who previously had been involved in a development deal with the Trump Organization in Batumi, Georgia.
Cohen stated that he spoke to Rtskhiladze in part because Rtskhiladze had pursued business ventures in Moscow, including a licensing deal with the Agalarov-owned Crocus Group.314 On September 22, 2015, Cohen forwarded a preliminary design study for the Trump Moscow project to Rtskhiladze, adding "I  look forward to your reply about this spectacular project in Moscow."
Rtskhiladze forwarded Cohen's email to an associate and wrote, "[i]f we could organize the meeting in New York at the highest level of the Russian Government and Mr. Trump this project would definitely receive the worldwide attention."

On September 24, 2015, Rtskhiladze sent Cohen an attachment that he described as a proposed "[l]etter to the Mayor of Moscow from Trump org," explaining that "[w]e need to send this letter to the Mayor of Moscow (second guy in Russia) he is aware of the potential project and will pledge his support."
In a second email to Cohen sent the same day, Rtskhiladze provided a translation of the letter, which described the Trump Moscow project as a  "symbol of stronger economic, business and cultural relationships between New York and Moscow and therefore United States and the Russian Federation."
On September 27, 2015, Rtskhiladze sent another email to Cohen, proposing that the Trump Organization partner on the Trump Moscow project with "Global Development Group LLC," which he described as being controlled by Michail Posikhin, a Russian architect, and Simon Nizharadze.
Cohen told the Office that he ultimately declined the proposal and instead continued to work with LC. Expert, the company represented by Felix Sater.

\paragraph{Letter of Intent and Contacts to Russian Government (October 2015-January 2016)}

\subparagraph{Trump Signs the Letter of Intent on behalf of the Trump Organization}

\subparagraph{Post-LOI Contacts with Individuals in Russia}

\paragraph{Discussions about Russia Travel by Michael Cohen or Candidate Trump (December 2015-June 2016)}

\subparagraph{Sater's Overtures to Cohen to Travel to Russia}

\subparagraph{Candidate Trump's Opportunities to Travel to Russia}

\subsubsection{George Papadopoulos}

\paragraph{Origins of Campaign Work}

\paragraph{Initial Russia-Related Contacts}

\paragraph{March 31 Foreign Policy Team Meeting}

\paragraph{George Papadopoulos Learns That Russia Has "Dirt" in the Form of Clinton Emails}

\paragraph{Russia-Related Communications With The Campaign}

\paragraph{Trump Campaign Knowledge of "Dirt"}

\paragraph{Additional George Papadopoulos Contact}

\subsubsection{Carter Page}

\paragraph{Background}

\paragraph{Origins of and Early Campaign Work}

\paragraph{Carter Page's July 2016 Trip To Moscow}

\paragraph{Later Campaign Work and Removal from the Campaign}

\subsubsection{Dimitri Simes and the Center for the National Interest}

\paragraph{CNI and Dimitri Simes Connect with the Trump Campaign}

\paragraph{National Interest Hosts a Foreign Policy Speech at the Mayflower Hotel}

\paragraph{Jeff Sessions's Post-Speech Interactions with CNI}

\paragraph{Jared Kushner's Continuing Contacts with Simes}

\subsubsection{June 9, 2016 Meeting at Trump Tower}

\paragraph{Setting Up the June 9 Meeting}

\subparagraph{Outreach to Donald Trump Jr}

\subparagraph{Awareness of the Meeting Within the Campaign}

\paragraph{The Events of June 9, 2016}

\subparagraph{Arrangements for the Meeting}

\subparagraph{Conduct of the Meeting}

\paragraph{Post-June 9 Events}

\subsubsection{Events at the Republican National Convention}

\paragraph{Ambassador Kislyak's Encounters with Senator Sessions and J.D. Gordon the Week of the RNC}

\paragraph{Change to Republican Party Platform}

\subsubsection{Post-Convention Contacts with Kislyak}

\paragraph{Ambassador Kislyak Invites J.D. Gordon to Breakfast at the Ambassador's Residence}

\paragraph{Senator Sessions's September 2016 Meeting with Ambassador Kislyak}

\subsubsection{Paul Manafort}

\paragraph{Paul Manafort's Ties to Russia and Ukraine}

\subparagraph{Oleg Deripaska Consulting Work}

\subparagraph{Political Consulting Work}

\subparagraph{Konstantin Kilimnik}

\paragraph{Contacts during Paul Manafort's Time with the Trump Campaign}

\subparagraph{Paul Manafort Joins the Campaign}

\subparagraph{Paul Manafort's Campaign-Period Contacts}

\subparagraph{Paul Manafort's Two Campaign-Period Meetings with Konstantin Kilimnik in the United States}

\paragraph{Post-Resignation Activities}

\subsection{Post-Election and Transition-Period Contacts}

\subsubsection{Immediate Post-Election Activity}

\paragraph{Outreach from the Russian Government}

\paragraph{High-Level Encouragement of Contacts through Alternative Channels}

\subsubsection{Kirill Dmitriev's Transition-Era Outreach to the Incoming Administration}

\paragraph{Background}

\paragraph{Kirill Dmitriev's Post-Election Contacts With the Incoming Administration}

\paragraph{Erik Prince and Kirill Dmitriev Meet in the Seychelles}

\subparagraph{George Nader and Erik Prince Arrange Seychelles Meeting with Dmitriev}

\subparagraph{The Seychelles Meetings}

\subparagraph{Erik Prince's Meeting with Steve Bannon after the Seychelles Trip}

\paragraph{Kirill Dmitriev's Post-Election Contact with Rick Gerson Regarding U.S.-Russia Relations}

\subsubsection{Ambassador Kislyak's Meeting with Jared Kushner and Michael Flynn in Trump Tower Following the Election}

\subsubsection{Jared Kushner's Meeting with Sergey Gorkov}

\subsubsection{Petr Aven's Outreach Efforts to the Transition Team}

\subsubsection{Carter Page Contact with Deputy Prime Minister Arkady Dvorkovich}

\subsubsection{Contacts With and Through Michael T. Flynn}

\paragraph{United Nations Vote on Israeli Settlements}

\paragraph{U.S. Sanctions Against Russia}
