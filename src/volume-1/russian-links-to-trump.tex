\section{Russian Government Links to and Contacts with The Trump Campaign}

The Office identified multiple contacts - "links," in the words of the Appointment Order - between Trump Campaign officials and individuals with ties to the Russian government.
The Office investigated whether those contacts constituted a third avenue of attempted Russian interference with or influence on the 2016 presidential election.
In particular, the investigation examined whether these contacts involved or resulted in coordination or a conspiracy with the Trump Campaign and Russia, including with respect to Russia providing assistance to the Campaign in exchange for any sort of favorable treatment in the future.
Based on the available information, the investigation did not establish such coordination.

This Section describes the principal links between the Trump Campaign and individuals with ties to the Russian government, including some contacts with Campaign officials or associates that have been publicly reported to involve Russian contacts.
Each subsection begins with an overview of the Russian contact at issue and then describes in detail the relevant facts, which are generally presented in chronological order, beginning with the early months of the Campaign and extending through the post-election, transition period.

\subsection{Campaign Period (September 2015 - November 8, 2016)}

Russian-government-connected individuals and media entities began showing interest in Trump's campaign in the months after he announced his candidacy in June 2015.
Because Trump's status as a public figure at the time was attributable in large part to his prior business and entertainment dealings, this Office investigated whether a business contact with Russia-linked individuals and entities during the campaign period -- the Trump Tower Moscow project, see Volume I, Section IV.A.1, infra -- led to or involved coordination of election assistance.

Outreach from individuals with ties to Russia continued in the spring and summer of 2016, when Trump was moving toward -- and eventually becoming -- the Republican nominee for President.
As set forth below, the Office also evaluated a series of links during this period: outreach to two of Trump's then-recently named foreign policy advisors, including a representation that Russia had "dirt" on Clinton in the form of thousands of emails (VolumeI, Sections IV.A.2 \& IV.A.3); dealings with a D.C.-based think tank that specializes in Russia and has connections with its government (Volume I, Section IV.A.4); a meeting at Trump Tower between the Campaign and a Russian lawyer promising dirt on candidate Clinton that was "part of Russia and its government's support for [Trump]" (Volume I, Section IV.A.5); events at the Republican National Convention (Volume I, Section IV.A.6); post-Convention contacts between Trump Campaign officials and Russia's ambassador to the United States (Volume I, Section IV.A.7); and contacts through campaign chairman Paul Manafort, who had previously worked for a Russian oligarch and a pro-Russian political party in Ukraine (Volume I, Section IV.A.8).

\subsubsection{Trump Tower Moscow Project}

The Trump Organization has pursued and completed projects outside the United States as part of its real estate portfolio.
Some projects have involved the acquisition and ownership (through subsidiary corporate structures) of property.
In other cases, the Trump Organization has executed licensing deals with real estate developers and management companies, often local to the country where the project was located.

Between at least 2013 and 2016, the Trump Organization explored a  similar licensing deal in Russia involving the construction of a Trump-branded property in Moscow.
The project, commonly referred to as a "Trump Tower Moscow" or "Trump Moscow" project,  anticipated a combination of commercial, hotel, and residential properties all within the same building.
Between 2013 and June 2016, several employees of the Trump Organization, including then-president of the organization Donald J. Trump, pursued a  Moscow deal with several Russian counterparties.
From the fall of 2015 until the middle of 2016, Michael Cohen spearheaded the Trump Organization's pursuit of a  Trump Tower Moscow project, including by reporting on the project's status to candidate Trump and other executives in the Trump Organization.

\paragraph{Trump Tower Moscow Venture with the Crocus Group (2013-2014)}

The Trump Organization and the Crocus Group, a Russian real estate conglomerate owned and controlled by Aras Agalarov, began discussing a Russia-based real estate project shortly after the conclusion of the 2013 Miss Universe pageant in Moscow.
Donald J. Trump Jr. served as the primary negotiator on behalf of the Trump Organization; Emin Agalarov (son of Aras Agalarov) and Irakli "Ike" Kaveladze represented the Crocus Group during negotiations, with the occasional assistance of Robe1t Goldstone.

In December 2013, Kaveladze and Trump Jr. negotiated and signed preliminary terms of an agreement for the Trump Tower Moscow project.
On December 23, 2013, after discussions with Donald J. Trump, the Trump Organization agreed to accept an arrangement whereby the organization received a flat 3.5\% commission on all sales, with no licensing fees or incentives.
The parties negotiated a letter of intent during January and February 2014.

From January 2014 through November 2014, the Trump Organization and Crocus Group discussed development plans for the Moscow project.
Some time before January 24, 2014, the Crocus Group sent the Trump Organization a proposal for a 800-unit, 194-meter building to be constructed at an Agalarov-owned site in Moscow called "Crocus City," which had also been the site of the Miss Universe pageant.
In February 2014, Ivanka Trump met with Emin Agalarov and toured the Crocus City site during a visit to Moscow.
From March 2014 through July 2014, the groups discussed "design standards" and other architectural elements.
For example, in July 2014, members of the Trump Organization sent Crocus Group counterparties questions about the "demographics of these prospective buyers" in the Crocus City area, the development of neighboring parcels in Crocus City, and concepts for redesigning portions of the building.
In August 2014, the Trump Organization requested specifications for a competing Marriott-branded tower being built in Crocus City.

Beginning in September 2014, the Trump Organization stopped responding in a timely fashion to correspondence and proposals from the Crocus Group.
Communications between the two groups continued through November 2014 with decreasing frequency; what appears to be the last communication is dated November 24, 2014.
The project appears not to have developed past the planning stage, and no construction occurred.

\paragraph{Communications with I.C. Expert Investment Company and Giorgi Rtskhiladze (Summer and Fall 2015)}

In the late summer of 2015, the Trump Organization received a new inquiry about pursuing a Trump Tower project in Moscow.
In approximately September 2015, Felix Sater, a New York-based real estate advisor, contacted Michael Cohen, then-executive vice president of the Trump Organization and special counsel to Donald J. Trump.
Sater had previously worked with the Trump Organization and advised it on a number of domestic and international projects.
Sater had explored the possibility of a Trump Tower project in Moscow while working with the Trump Organiz ation and therefore knew of the organization's general interest in completing a  deal there.
Sater had also served as an informal agent of the Trump Organization in Moscow previously  and had accompanied lvanka Trump and Donald Trump Jr. to Moscow in the mid-2000s.

Sater contacted Cohen on behalf of I.C. Expert Investment Company (I.C. Expert), a Russian real-estate development corporation controlled by Andrei Vladimirovich Rozov.
Sater had known Rozov since approximately 2007 and, in 2014, had served as an agent on behalf of Rozov during Rozov's purchase of a building in New York City.
Sater later contacted Rozov and proposed that I.C. Expert pursue a Trump Tower Moscow project in which I.C. Expert would license the name and brand from the Trump Organization but construct the building on its own. Sater worked on the deal with Rozov and another.employee of I.C. Expert.

Cohen was the only Trump Organization representative to negotiate directly with I.C. Expert or its agents.
In approximately September 2015, Cohen obtained approval to negotiate with I.C. Expert from candidate Trump, who was then president of the Trump Organization.
Cohen provided updates directly to Trump about the project throughout 2015 and into 2016, assuring him the project was continuing.
Cohen also discussed the Trump Moscow project with Ivanka Trump as to design elements (such as possible architects to use for the project) and Donald J. Trump Jr. (about his experience in Moscow and possible involvement in the project) during the fall of 2015.

Also during the fall of 2015, Cohen communicated about the Trump Moscow proposal with Giorgi Rtskhiladze, a business executive who previously had been involved in a development deal with the Trump Organization in Batumi, Georgia.
Cohen stated that he spoke to Rtskhiladze in part because Rtskhiladze had pursued business ventures in Moscow, including a licensing deal with the Agalarov-owned Crocus Group.314 On September 22, 2015, Cohen forwarded a preliminary design study for the Trump Moscow project to Rtskhiladze, adding "I  look forward to your reply about this spectacular project in Moscow."
Rtskhiladze forwarded Cohen's email to an associate and wrote, "[i]f we could organize the meeting in New York at the highest level of the Russian Government and Mr. Trump this project would definitely receive the worldwide attention."

On September 24, 2015, Rtskhiladze sent Cohen an attachment that he described as a proposed "[l]etter to the Mayor of Moscow from Trump org," explaining that "[w]e need to send this letter to the Mayor of Moscow (second guy in Russia) he is aware of the potential project and will pledge his support."
In a second email to Cohen sent the same day, Rtskhiladze provided a translation of the letter, which described the Trump Moscow project as a  "symbol of stronger economic, business and cultural relationships between New York and Moscow and therefore United States and the Russian Federation."
On September 27, 2015, Rtskhiladze sent another email to Cohen, proposing that the Trump Organization partner on the Trump Moscow project with "Global Development Group LLC," which he described as being controlled by Michail Posikhin, a Russian architect, and Simon Nizharadze.
Cohen told the Office that he ultimately declined the proposal and instead continued to work with LC. Expert, the company represented by Felix Sater.

\paragraph{Letter of Intent and Contacts to Russian Government (October 2015-January 2016)}

\subparagraph{Trump Signs the Letter of Intent on behalf of the Trump Organization}

Between approximately October 13, 2015 and November 2, 2015, the Trump Organization (through its subsidiary Trump Acquisition, LLC) and I.C. Expert completed a letter of intent (LOI) for a Trump Moscow property.
The LOI, signed by Trump for the Trump Organization and Rozov on behalf of I.C. Expert, was "intended to facilitate further discussions" in order to "attempt to enter into a  mutually acceptable agreement" related to the Trump-branded project in Moscow.
The LOI contemplated a development with residential, hotel, commercial, and office components, and called for"[a]pproximately 250 first class, luxury residential condominiums," as well as "[o]ne first class, luxury hotel consisting of approximately 15 floors and containing not fewer than 150 hotel rooms."
For the residential and commercial portions of the project, the Trump Organization would receive between 1\% and 5\% of all condominium sales, plus 3\% of all rental and other revenue.
For the project's hotel portion, the Trump Organization would receive a base fee of 3\% of gross operating revenues for the first five years and 4\% thereafter, plus a  separate incentive fee of 20\% of operating profit.
Under the LOI, the Trump Organization also would receive a \$4 million "up-front fee" prior to groundbreaking.
Under these terms, the Trump Organization stood to earn substantial sums over the lifetime of the project, without assuming significant liabilities or financing commitments.

On November 3, 2015, the day after the Trump Organization transmitted the LOI, Sater emailed Cohen suggesting that the Trump Moscow project could be used to increase candidate Trump's chances at being elected, writing:

\begin{quote}
Buddy our boy can become President of the USA and we can engineer it.
I will get all of Putins team to buy in on this, I will manage this process ....
Michael, Putin gets on stage with Donald for a ribbon cutting for Trump Moscow, and Donald owns the republican nomination.
And possibly beats Hillary and our boy is in....
We will manage this process better than anyone.
You and I will get Donald and Vladimir on a stage together very shortly.
That the game changer.
\end{quote}

Later that day, Sater followed up:

\begin{quote}
Donald doesn't stare down, he negotiates and understands the economic issues and Putin only want to deal with a pragmatic leader, and a successful business man is a good candidate for someone who knows how to negotiate.
"Business, politics, whatever it all is the same for someone who knows how to deal"
I think I can get Putin to say that at the Trump Moscow press conference.
If he says it we own this election.
Americas most difficult adversary agreeing that Donald is a good guy to negotiate ....
We can own this election.
Michael my next steps are very sensitive with Putins very very close people, we can pull this off.
Michael lets go.
2 boys from Brooklyn getting a USA president elected.
This is good really good.
\end{quote}

According to Cohen, he did not consider the political import of the Trump Moscow project to the 2016 U.S. presidential election at the time.
Cohen also did not recall candidate Trump or anyone affiliated with the Trump Campaign discussing the political implications of the Trump Moscow project with him.
However, Cohen recalled conversations with Trump in which the candidate suggested that his campaign would be a significant "infomercial" for Trump-branded properties.

\subparagraph{Post-LOI Contacts with Individuals in Russia}

Given the size of the Trump Moscow project, Sater and Cohen believed the project required approval (whether express or implicit) from the Russian national government, including from the Presidential Administration of Russia.
Sater stated that he th erefore began to contact the Presidential Administration through another Russian business contact.
In early negotiations with the Trump Organization, Sater had alluded to the need for government approval and his attempts to set up meetings with Russian officials.
On October 12, 2015, for example, Sater wrote to Cohen that "all we need is Putin on board and we are golden," and that a "meeting with Putin and top deputy is tentatively set for the 14th [of October]."
\blackout{Grand Jury} this meeting was being coordinated by associates in Russia and that he had no direct interaction with the Russian government.

Approximately a month later, after the LOI had been signed, Lana Erchova emailed lvanka Trump on behalf of Erchova's then-husband Dmitry Klokov, to offer Klokov's assistance to the Trump Campaign.
Klokov was at that time Director of External Communications for PJSC Federal Grid Company of Unified Energy System, a large Russian electricity transmission company, and had been previously employed as an aide and press secretary to Russia's energy minister.
Ivanka Trump forwarded the email to Cohen.
He told the Office that, after receiving this inquiry, he had conducted an internet search for Klokov's name and concluded (incorrectly) that Klokov was a former Olympic weightlifter.

Between November 18 and 19, 2015, Klokov and Cohen had at least one telephone call and exchanged several emails.
Describing himself in emails to Cohen as a "trusted person" who could offer the Campaign "political synergy" and "synergy on a government level," Klokov recommended that Cohen travel to Russia to speak with him and an unidentified intermediary.
Klokov said that those conversations could facilitate a later meeting in Russia between the candidate and an individual Klokov described as "our person of interest."
In an email to the Office, Erchova later identified the "person of interest" as Russian President Vladimir Putin.

In the telephone call and follow-on emails with Klokov, Cohen discussed his desire to use a near-term trip to Russia to do site surveys and talk over the Trump Moscow project with local developers.
Cohen registered his willingness also to meet with Klokov and the unidentified intermediary, but was emphatic that all meetings in Russia involving him or candidate Trump -- including a possible meeting between candidate Trump and Putin-would need to be "in conjunction with the development and an official visit" with the Trump Organization receiving a formal invitation to visit.
(Klokov had written previously that "the visit [by candidate Trump to Russia] has to be informal.")

Klokov had also previously recommended to Cohen that he separate their negotiations over a possible meeting between Trump and "the person of interest" from any existing business track.
Re-emphasizing that his outreach was not done on behalf of any business, Klokov added in second email to Cohen that, if publicized well, such a meeting could have "phenomenal" impact "in a business dimension" and that the "person of interest['s]" "most important support" could have significant ramifications for the "level of projects and their capacity."
Klokov concluded by telling Cohen that there was "no bigger warranty in any project than [the] consent of the person of interest."
Cohen rejected the proposal, saying that "[c]urrently our LOI developer is in talks with VP's Chief of Staff and arranging a formal invite for the two to meet."
This email appears to be their final exchange, and the investigation did not identify evidence that Cohen brought Klokov' s initial offer of assistance to the Campaign's attention or that anyone associated with the Trump Organization or the Campaign dealt with Klokov at a later date.
Cohen explained that he did not pursue the proposed meeting because he was already working on the Moscow Project with Sater, who Cohen understood to have his own connections to the Russian government.

By late December 2015, however, Cohen was complaining that Sater had not been able to use those connections to set up the promised meeting with Russian government officials.
Cohen told Sater that he was "setting up the meeting myself."% 345
On January 11, 2016,  Cohen emailed the office of Dmitry Peskov, the Russian government's press secretary, indicating that he desired contact with Sergei Ivanov,  Putin's chief of staff.
Cohen erroneously used the email address "Pr_peskova@prpress.gof.ru" instead of "Pr_peskova@prpress.gov.ru," so the email apparently did not go through.% 346
On January 14, 2016, Cohen emailed a different address (info@prpress.gov.ru) with the following message:

\begin{quote}
Dear Mr. Peskov,

Over the past few months, I have been working with a company based in Russia regarding the development of a Trump Tower-Moscow project in Moscow City.
Without getting into lengthy specifics, the communication between our two sides has stalled.
As this project is too important, I am hereby requesting your assistance.
I respectfully request someone, preferably you; contact me so that I might discuss the specifics as well as arranging meetings with the appropriate individuals.
I thank you in advance for your assistance and look forward to hearing from you soon.% 347
\end{quote}

Two days later, Cohen sent an email to Pr_peskova@prpress.gov.ru, repeating his request to speak with Sergei Ivanov.% 348

Cohen testified to Congress, and initially told the Office, that he did not recall receiving a response to this email inquiry and that he decided to terminate any further work on the Trump Moscow project as of January 2016.
Cohen later admitted that these statements were false.
In fact, Cohen had received (and recalled receiving) a  response to his inquiry, and he continued to work on and update candidate Trump on the project through as late as June 2016.% 349

On January 20, 2016, Cohen received an email from Elena Poliakova, Peskov's personal assistant.
Writing from her personal email account, Poliakova stated that she had been trying to reach Cohen and asked that he call her on the personal number that she provided.% 350
Shortly after receiving Poliakova's email, Cohen called and spoke to her for 20 minutes.% 351
Cohen described to Poliakova his position at the Trump Organization and outlined the proposed Trump Moscow project, including information about the Russian counterparty with which the Trump Organization had partnered.
Cohen requested assistance in moving the project forward, both in securing land to build the project and with financing.
According to Cohen, Poliakova asked detailed questions and took notes, stating that she would need to follow up with others in Russia.% 352

Cohen could not recall any direct follow-up from Poliakova or from any other representative of the Russian government, nor did the Office identify any evidence of direct follow-up.
However, the day after Cohen's call with Poliakova, Sater texted Cohen, asking him to "[c]all me when you have a few minutes to chat ... It's about Putin they called today."% 353
Sater then sent a  draft invitation for Cohen to visit Moscow to discuss the Trump Moscow project, % 354
along with a note to "[t]ell me if the letter is good as amended by me or make whatever changes you want and send it  back to me."% 355
After a  further round of edits, on January 25, 2016,  Sater sent Cohen an invitation-signed by Andrey Ryabinskiy of the company MHJ-to travel to "Moscow for a working visit" about the "prospects of development and the construction business in Russia," "the various land plots available suited for construction of this enormous Tower," and "the opportunity to co-ordinate a follow up visit to Moscow by Mr. Donald Trump."% 356
According to Cohen,  he elected not to travel at the time because of concerns about the lack of concrete proposals about land plots that could be considered as options for the project.% 357

\paragraph{Discussions about Russia Travel by Michael Cohen or Candidate Trump (December 2015-June 2016)}

\subparagraph{Sater's Overtures to Cohen to Travel to Russia}

The late January communication was neither the first nor the last time that Cohen contemplated visiting Russia in pursuit of the Trump Moscow project.
Beginning in late 2015, Sater repeatedly tried to arrange for Cohen and candidate Trump, as representatives of the Trump Organization, to travel to Russia to meet with Russian government officials and possible financing partners.
In December 2015, Sater sent Cohen a number of emails about logistics for traveling to Russia for meetings.% 358
On December 19, 2015, Sater wrote:

\begin{quote}
Please call me I  have Evgeney [Dvoskin] on the other line.% 359
He needs a copy of your and Donald's passports they need a scan of every page of the passports.
Invitations & Visas will be issued this week by VTB Bank to discuss financing for Trump Tower Moscow. Politically neither Putins office nor Ministry of Foreign Affairs cannot issue invite, so they are inviting commercially/ business.
VTB is Russia's 2 biggest bank and VTB Bank CEO Andrey Kostin, will be at all meetings with Putin so that it is a business meeting not political.
We will be invited to Russian consulate this week to receive invite & have visa issued.% 360
\end{quote}

In response, Cohen texted Sater an image of his own passport.% 361
Cohen told the Office that at one point he requested a copy of candidate Trump's passport from Rhona Graff, Trump's executive assistant at the Trump Organization, and that Graff later brought Trump's passport to Cohen's office.% 362
The investigation did not, however, establish that the passport was forwarded to Sater.% 363

Into the spring of 2016, Sater and Cohen continued to discuss a trip to Moscow in connection with the Trump Moscow project.
On April 20, 2016, Sater wrote Cohen, "[t]he People wanted to know when you are coming?"% 364
On May 4, 2016, Sater followed up:

\begin{quote}
I had a chat with Moscow.
ASSUMING the trip does happen the question is before or after the convention.
I said I believe, but don't know for sure, that's it's probably after the convention.
Obviously the pre-meeting trip (you only) can happen anytime you want but the 2 big guys where [sic] the question.
I said I would confirm and revert....
Let me know about If I was right by saying I believe after Cleveland and also when you want to speak to them and possibly fly over.% 365
\end{quote}

Cohen responded, "My trip before Cleveland.
Trump once he becomes the nominee after the convention."% 366

The day after this exchange, Sater tied Cohen's travel to Russia to the St. Petersburg International Economic Forum ("Forum"), an annual event attended by prominent Russian politicians and businessmen.
Sater told the Office that he was informed by a business associate that Peskov wanted to invite Cohen to the Forum.% 367
On May 5, 2016, Sater wrote to Cohen:

\begin{quote}
Peskov would like to invite you as his guest to the St. Petersburg Forum which is Russia's Davos it's June 16-19.
He wants to meet there with you and possibly introduce you to either Putin or Medvedev, as they are not sure if 1 or both will be there.
This is perfect.
The entire business class of Russia will be there as well.
He said anything you want to discuss including dates and subjects are on the table to discuss[.]% 368
\end{quote}

The following day, Sater asked Cohen to confirm those dates would work for him to travel; Cohen wrote back, "[w]orks for me."% 369

On June 9, 2016, Sater sent Cohen a notice that he (Sater) was completing the badges for the Forum, adding, "Putin is there on the 17th very strong chance you will meet him as well."% 370
On June 13, 2016, Sater forwarded Cohen an invitation to the Forum signed by the Director of the Roscongress Foundation, the Russian entity organizing the Forum.% 371
Sater also sent Cohen a Russian visa application and asked him to send two passport photos.% 372
According to Cohen, the invitation gave no indication that Peskov had been involved in inviting him.
Cohen was concerned that Russian officials were not actually involved or were not interested in meeting with him (as Sater had alleged), and so he decided not to go to the Forum.% 373
On June 14, 2016, Cohen met Sater in the lobby of the Trump Tower in New York and informed him that he would not be traveling at that time.% 374

\subparagraph{Candidate Trump's Opportunities to Travel to Russia}

The investigation identified evidence that, during the period the Trump Moscow project was under consideration, the possibility of candidate Trump visiting Russia arose in two contexts.

First, in interviews with the Office, Cohen stated that he discussed the subject of traveling to Russia with Trump twice: once in late 2015; and again in spring 2016.% 375
According to Cohen, Trump indicated a willingness to travel if it would assist the project significantly.
On one occasion, Trump told Cohen to speak with then-campaign manager Corey Lewandowski to coordinate the candidate's schedule.
Cohen recalled that he spoke with Lewandowski, who suggested that they speak again when Cohen had actual dates to evaluate.
Cohen indicated, however, that he knew that travel prior to the Republican National Convention would be impossible given the candidate's preexisting commitments to the Campaign.% 376

Second, like Cohen, Trump received and turned down an invitation to the St. Petersburg International Economic Forum.
In late December 2015, Mira Duma - a contact of Ivanka Trump's from the fashion industry - first passed along invitations for Ivanka Trump and candidate Trump from Sergei Prikhodko, a Deputy Prime Minister of the Russian Federation.% 377
On January 14, 2016, Rhona Graff sent an email to Duma stating that Trump was "honored to be asked to participate in the highly prestigious" Forum event, but that he would "have to decline" the invitation given his "very grueling and full travel schedule" as a presidential candidate.% 378
Graff asked Duma whether she recommended that Graff "send a formal note to the Deputy Prime Minister" declining his invitation; Duma replied that a formal note would be "great."% 379

It does not appear that Graff prepared that note immediately.
According to written answers from President Trump,% 380
Graff received an email from Deputy Prime Minister Prikhodko on March 17, 2016, again inviting Trump to participate in the 2016 Forum in St. Petersburg.% 381
Two weeks later, on March 31, 2016, Graff prepared for Trump's signature a two-paragraph letter declining the invitation.% 382
The letter stated that Trump's "schedule has become extremely demanding" because of the presidential campaign, that he "already ha[d] several commitments in the United States" for the time of the Forum, but that he otherwise "would have gladly given every consideration to attending such an important event."% 383
Graff forwarded the letter to another executive assistant at the Trump Organization with instructions to print the document on letterhead for Trump to sign.% 384

At approximately the same time that the letter was being prepared, Robert Foresman-a New York-based investment banker-began reaching out to Graff to secure an in-person meeting with candidate Trump.
According to Foresman, he had been asked by Anton Kobyakov, a Russian presidential aide involved with the Roscongress Foundation, to see if Trump could speak at the Forum.% 385
Foresman first emailed Graff on March 31, 2016, following a phone introduction brokered through Trump business associate Mark Burnett (who produced the te levision show The Apprentice).
In his email, Foresman referenced his long-standing personal and professional expertise in Russia and Ukraine, his work setting up an early "private channel" between Vladimir Putin and former U.S. President George W. Bush, and an "approach" he had received from "senior Kremlin officials" about the candidate.
Foresman asked Graff for a meeting with the candidate, Corey Lewandowski, or "another relevant person" to discuss this and other "concrete things" Foresman felt uncomfortable discussing over "unsecure email."% 386
On April 4, 2016, Graff forwarded Foresman' s meeting request to Jessica Macchia, another executive assistant to Trump.% 387

With no response forthcoming, Foresman twice sent reminders to Graff - first on April 26 and again on April 30, 2016.% 388
Graff sent an apology to Foresman and forwarded his April 26 email (as well as his initial March 2016 email) to Lewandowski.% 389
On May 2, 2016, Graff forwarded Foresman's April 30 email-which suggested an alternative meeting with Donald Trump Jr. or Eric Trump so that Foresman could convey to them information that "should be conveyed to [the candidate] personally or [to] someone [the candidate] absolutely trusts" - to policy advisor Stephen Miller.% 390

No communications or other evidence obtained by the Office indicate that the Trump Campaign learned that Foresman was reaching out to invite the candidate to the Forum or that the Campaign otherwise followed up with Foresman until after the election, when he interacted with the Transition Team as he pursued a  possible position in the incoming Administration.% 391
When interviewed by the Office, Foresman denied that the specific "approach" from "senior Kremlin officials" noted in his March 31, 2016 email was anything other than Kobyakov's invitation to Roscongress.
According to Foresman, the "concrete things" he referenced in the same email were a combination of the invitation itself, Foresman's personal perspectives on the invitation and Russia policy in general, and details of a Ukraine plan supported by a U.S. think tank (EastWest Institute).
Foresman told the Office that Kobyakov had extended similar invitations through him to another Republican presidential candidate and one other politician.
Foresman also said that Kobyakov had asked Foresman to invite Trump to speak after that other presidential candidate withdrew from the race and the other politician's participation did not work out.% 392
Finally, Foresman claimed to have no plans to establish a back channel involving Trump, stating the reference to his involvement in the Bush-Putin back channel was meant to burnish his credentials to the Campaign.
Foresman commented that he had not recognized any of the experts announced as Trump's foreign policy team in March 2016, and wanted to secure an in-person meeting with the candidate to share his professional background and policy views, including that Trump should decline Kobyakov's invitation to speak at the Forum.% 393

\subsubsection{George Papadopoulos}

\paragraph{Origins of Campaign Work}

\paragraph{Initial Russia-Related Contacts}

\paragraph{March 31 Foreign Policy Team Meeting}

\paragraph{George Papadopoulos Learns That Russia Has "Dirt" in the Form of Clinton Emails}

\paragraph{Russia-Related Communications With The Campaign}

\paragraph{Trump Campaign Knowledge of "Dirt"}

\paragraph{Additional George Papadopoulos Contact}

\subsubsection{Carter Page}

\paragraph{Background}

\paragraph{Origins of and Early Campaign Work}

\paragraph{Carter Page's July 2016 Trip To Moscow}

\paragraph{Later Campaign Work and Removal from the Campaign}

\subsubsection{Dimitri Simes and the Center for the National Interest}

\paragraph{CNI and Dimitri Simes Connect with the Trump Campaign}

\paragraph{National Interest Hosts a Foreign Policy Speech at the Mayflower Hotel}

\paragraph{Jeff Sessions's Post-Speech Interactions with CNI}

\paragraph{Jared Kushner's Continuing Contacts with Simes}

\subsubsection{June 9, 2016 Meeting at Trump Tower}

\paragraph{Setting Up the June 9 Meeting}

\subparagraph{Outreach to Donald Trump Jr}

\subparagraph{Awareness of the Meeting Within the Campaign}

\paragraph{The Events of June 9, 2016}

\subparagraph{Arrangements for the Meeting}

\subparagraph{Conduct of the Meeting}

\paragraph{Post-June 9 Events}

\subsubsection{Events at the Republican National Convention}

\paragraph{Ambassador Kislyak's Encounters with Senator Sessions and J.D. Gordon the Week of the RNC}

\paragraph{Change to Republican Party Platform}

\subsubsection{Post-Convention Contacts with Kislyak}

\paragraph{Ambassador Kislyak Invites J.D. Gordon to Breakfast at the Ambassador's Residence}

\paragraph{Senator Sessions's September 2016 Meeting with Ambassador Kislyak}

\subsubsection{Paul Manafort}

\paragraph{Paul Manafort's Ties to Russia and Ukraine}

\subparagraph{Oleg Deripaska Consulting Work}

\subparagraph{Political Consulting Work}

\subparagraph{Konstantin Kilimnik}

\paragraph{Contacts during Paul Manafort's Time with the Trump Campaign}

\subparagraph{Paul Manafort Joins the Campaign}

\subparagraph{Paul Manafort's Campaign-Period Contacts}

\subparagraph{Paul Manafort's Two Campaign-Period Meetings with Konstantin Kilimnik in the United States}

\paragraph{Post-Resignation Activities}

\subsection{Post-Election and Transition-Period Contacts}

\subsubsection{Immediate Post-Election Activity}

\paragraph{Outreach from the Russian Government}

\paragraph{High-Level Encouragement of Contacts through Alternative Channels}

\subsubsection{Kirill Dmitriev's Transition-Era Outreach to the Incoming Administration}

\paragraph{Background}

\paragraph{Kirill Dmitriev's Post-Election Contacts With the Incoming Administration}

\paragraph{Erik Prince and Kirill Dmitriev Meet in the Seychelles}

\subparagraph{George Nader and Erik Prince Arrange Seychelles Meeting with Dmitriev}

\subparagraph{The Seychelles Meetings}

\subparagraph{Erik Prince's Meeting with Steve Bannon after the Seychelles Trip}

\paragraph{Kirill Dmitriev's Post-Election Contact with Rick Gerson Regarding U.S.-Russia Relations}

\subsubsection{Ambassador Kislyak's Meeting with Jared Kushner and Michael Flynn in Trump Tower Following the Election}

\subsubsection{Jared Kushner's Meeting with Sergey Gorkov}

\subsubsection{Petr Aven's Outreach Efforts to the Transition Team}

\subsubsection{Carter Page Contact with Deputy Prime Minister Arkady Dvorkovich}

\subsubsection{Contacts With and Through Michael T. Flynn}

\paragraph{United Nations Vote on Israeli Settlements}

\paragraph{U.S. Sanctions Against Russia}
