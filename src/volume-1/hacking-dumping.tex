\section{Russian Hacking and Dumping Operations}

Beginning in March 2016, units of the Russian Federation's Main Intelligence Directorate of the General Staff (GRU) hacked the computers and email accounts of organizations, employees, and volunteers supporting the Clinton Campaign, including the email account of campaign chairman John Podesta.
Starting in April 2016, the GRU hacked into the computer networks of the Democratic Congressional Campaign Committee (DCCC) and the Democratic National Committee (DNC).
The GRU targeted hundreds of email accounts used by Clinton Campaign employees, advisors, and volunteers.
In total, the GRU stole hundreds of thousands of documents from the compromised email accounts and networks.% 109
\footnote{109}
The GRU later released stolen Clinton Campaign and DNC documents through online personas, "DCLeaks" and "Guccifer 2.0," and later through the organization WikiLeaks.
The release of the documents was designed and timed to interfere with the 2016 U.S. presidential election and undermine the Clinton Campaign.

The Trump Campaign showed interest in the WikiLeaks releases and, in the summer and fall of 2016, \xblackout{Harm to Ongoing Matter: Lorem ipsum dolor sit amet, consectetur adipiscing elit, sed do eiusmod tempor incididunt ut labore et dolore magna aliqua. Ut enim ad minim veniam, quis nostrud exercitation ullamco laboris nisi ut aliquip ex ea commodo consequat.}
WikiLeaks's first Clinton-related release \xblackout{HOM}, the Trump Campaign stayed in contact \xblackout{HOM} about WikiLeaks's activities.
The investigation was unable to resolve \xblackout{Harm to Ongoing Matter.}
WikiLeaks's release of the stolen Podesta emails on October 7, 2016, the same day a video from years earlier was published of Trump using graphic language about women.

\subsection{GRU Hacking Directed at the Clinton Campaign}

\subsubsection{GRU Units Target the Clinton Campaign}

Two military units of the GRU carried out the computer intrusions into the Clinton Campaign, DNC, and DCCC: Military Units 26165 and 74455.% 110
\footnote{110}
Military Unit 26165 is a GRU cyber unit dedicated to targeting military, political, governmental, and non-governmental organizations outside of Russia, including in the United States.% 111
\footnote{111}
The unit was sub-divided into departments with different specialties.
One department, for example, developed specialized malicious software "malware", while another department conducted large-scale spearphishing campaigns.% 112
\footnote{112}
\xblackout{Investigative Technique} a bitcoin mining operation to secure bitcoins used to purchase computer infrastructure used in hacking operations.% 113
\footnote{113}

Military Unit 74455 is a related GRU unit with multiple departments that engaged in cyber operations.
Unit 74455 assisted in the release of documents stolen by Unit 26165, the promotion of those releases, and the publication of anti-Clinton content on social media accounts operated by the GRU.
Officers from Unit 74455 separately hacked computers belonging to state boards of elections, secretaries of state, and U.S. companies that supplied software and other technology related to the administration of U.S. elections.% 114
\footnote{114}

Beginning in mid-March 2016, Unit 26165 had primary responsibility for hacking the DCCC and DNC, as well as email accounts of individuals affiliated with the Clinton Campaign:% 115
\footnote{115}

\begin{itemize}
    \item Unit 26165 used \xblackout{Investigative Technique} to learn about \xblackout{Investigative Technique} different Democratic websites, including democrats.org, hillaryclinton.com, dnc.org, and dccc.org.
    \xblackout{Investigative Technique: Lorem ipsum dolor sit amet, consectetur adipiscing elit, sed do eiusmod tempor incididunt ut labore et dolore magna aliqua.} began before the GRU had obtained any credentials or gained access to these networks,  indicating that the later DCCC and DNC intrusions were not crimes of opportunity but rather the result of targeting.% 116
    \footnote{116}
    \item GRU officers also sent hundreds of spearphishing emails to the work and personal email accounts of Clinton Campaign employees and volunteers.
    Between March 10, 2016 and March 15, 2016, Unit 26165 appears to have sent approximately 90 spearphishing emails to email accounts at hillaryclinton.com.
    Starting on March 15, 2016, the GRU began targeting Google email accounts used by Clinton Campaign employees, along with a smaller number of dnc.org email accounts.% 117
    \footnote{117}
\end{itemize}

The GRU spearphishing operation enabled it to gain access to numerous email accounts of Clinton Campaign employees and volunteers, including campaign chairman John Podesta, junior volunteers assigned to the Clinton Campaign's advance team, informal Clinton Campaign advisors, and a DNC employee.% 118
\footnote{118}
GRU officers stole tens of thousands of emails from spearphishing victims, including various Clinton Campaign-related communications.

\subsubsection{Intrusions into the DCCC and DNC Networks}

\paragraph{Initial Access}

By no later than April 12, 2016, the GRU had gained access to the DCCC computer network using the credentials stolen from a DCCC employee who had been successfully spearphished the week before.
Over the ensuing weeks, the GRU traversed the network, identifying different computers connected to the DCCC network.
By stealing network access credentials along the way (including those of IT administrators with unrestricted access to the system), the GRU compromised approximately 29 different computers on the DCCC network.% 119
\footnote{119}

Approximately six days after first hacking into the DCCC network, on April 18, 2016, GRU officers gained access to the DNC network via a virtual private network (VPN) connection% 120
\footnote{120}
between the DCCC and DNC networks.% 121
\footnote{121}
Between April 18, 2016 and June 8, 2016, Unit 26165 compromised more than 30 computers on the DNC network, including the DNC mail server and shared file server.% 122
\footnote{122}

\paragraph{Implantation of Malware on DCCC and DNC Networks}

Unit 26165 implanted on the DCCC and DNC networks two types of customized malware,% 123
\footnote{123}
known as "X-Agent" and "X-Tunnel";  Mimikatz, a credential-harvesting tool; and rar.exe, a tool used in these intrusions to compile and compress materials for exfiltration.
X-Agent was a multi-function hacking tool that allowed Unit 26165 to log keystrokes, take screenshots, and gather other data about the infected computers (e.g., file directories,  operating systems).% 124
\footnote{124}
X-Tunnel was a  hacking tool that created an encrypted connection between the victim DCCC/DNC computers and GRU-controlled computers outside the DCCC and DNC networks that was capable of large-scale data transfers.% 125
\footnote{125}
GRU officers then used X-Tunnel to exfiltrate stolen data from the victim computers.

To operate X-Agent and X-Tunnel on the DCCC and DNC networks, Unit 26165 officers set up a group of computers outside those networks to communicate with the implanted malware.% 126
\footnote{126}
The first set of GRU-controlled computers, known by the GRU as "middle servers," sent and received messages to and from malware on the DNC/DCCC networks.
The middle servers, in turn, relayed messages to a second set of GRU-controlled computers, labeled internally by the GRU as an "AMS Panel."
The AMS Panel \xblackout{Investigative Technique} served as a nerve center through which GRU officers monitored and directed the malware's operations on the DNC/DCCC networks.% 127
\footnote{127}

The AMS Panel used to control X-Agent during the DCCC and DNC intrusions was housed on a leased computer located near \xblackout{IT: Lorem} Arizona.% 128
\footnote{128}
\xblackout{Investigative Technique: Lorem ipsum dolor sit amet, consectetur adipiscing elit, sed do eiusmod tempor incididunt ut labore et dolore magna aliqua. Ut enim ad minim veniam, quis nostrud exercitation ullamco laboris nisi ut aliquip ex ea commodo consequat.}% 129
\footnote{129}

\xblackout{Harm to Ongoing Matter: Lorem ipsum dolor sit amet, consectetur adipiscing elit, sed do eiusmod tempor incididunt ut labore et dolore magna aliqua. Ut enim ad minim veniam, quis nostrud exercitation ullamco laboris nisi ut aliquip ex ea commodo consequat. Duis aute irure dolor in reprehenderit in voluptate velit esse cillum dolore eu fugiat nulla pariatur. Excepteur sint occaecat cupidatat non proident, sunt in culpa qui officia deserunt mollit anim id est laborum. Lorem ipsum dolor sit amet, consectetur adipiscing elit, sed do eiusmod tempor incididunt ut labore et dolore magna aliqua. Ut enim ad minim veniam, quis nostrud exercitation ullamco laboris nisi ut aliquip ex ea commodo consequat. Duis aute irure dolor in reprehenderit in voluptate velit esse cillum dolore eu fugiat nulla pariatur. Excepteur sint occaecat cupidatat non proident, sunt in culpa qui officia deserunt mollit anim id est laborum.}

\xblackout{Investigative Technique: Lorem ipsum dolor sit amet, consectetur adipiscing elit, sed do eiusmod tempor incididunt ut labore et dolore magna aliqua. Ut enim ad minim veniam, quis nostrud exercitation ullamco laboris nisi ut aliquip ex ea commodo consequat. Duis aute irure dolor in reprehenderit in voluptate velit esse cillum dolore eu fugiat nulla pariatur. Excepteur sint occaecat cupidatat non proident, sunt in culpa qui officia deserunt mollit anim id est laborum.}

The Arizona-based AMS Panel also stored thousands of files containing keylogging sessions captured through X-Agent.
These sessions were captured as GRU officers monitored DCCC and DNC employees' work on infected computers regularly between April 2016 and June 2016.
Data captured in these key logging sessions included passwords, internal communications between employees, banking information, and sensitive personal information.

\paragraph{Theft of Documents from DNC and DCCC Networks}

Officers from Unit 26165 stole thousands of documents from the DCCC and DNC networks, including significant amounts of data pertaining to the 2016 U.S. federal elections.
Stolen documents included internal strategy documents, fundraising data, opposition research, and emails from the work inboxes of DNC employees.% 130
\footnote{130}

The GRU began stealing DCCC data shortly after it gained access to the network.
On April 14, 2016 (approximately three days after the initial intrusion) GRU officers downloaded rar.exe onto the DCCC's document server.
The following day, the GRU searched one compromised DCCC computer for files containing search terms that included "Hillary," "DNC," "Cruz," and "Trump."% 131
\footnote{131}
On April 25, 2016, the GRU collected and compressed PDF and Microsoft documents from folders on the DCCC's shared file server that pertained to the 2016 election.% 132
\footnote{132}
The GRU appears to have compressed and exfiltrated over 70 gigabytes of data from this file server.% 133
\footnote{133}

The GRU also stole documents from the DNC network shortly after gaining access.
On April 22, 2016, the GRU copied files from the DNC network to GRU-controlled computers.
Stolen documents included the DNC's opposition research into candidate Trump.% 134
\footnote{134}
Between approximately May 25, 2016 and June 1, 2016, GRU officers accessed the DNC's mail server from a  GRU-controlled computer leased inside the United States.% 135
\footnote{135}
During these connections, Unit 26165 officers appear to have stolen thousands of emails and attachments, which were later released by WikiLeaks in July 2016.% 135
\footnote{135}

\subsection{Dissemination of the Hacked Materials}

