\section{Factual Results of the Obstruction Investigation}

\subsection{The Campaign's Response to Reports About Russian Support for Trump}

\subsubsection{Press Reports Allege Links Between the Trump Campaign and Russia}

\subsubsection{The Trump Campaign Reacts to WikiLeaks's Release of Hacked Emails}

\subsubsection{The Trump Campaign Reacts to Allegations That Russia was Seeking to Aid Candidate Trump}

\subsubsection{After the Election, Trump Continues to Deny Any Contacts or Connections with Russia or That Russia Aided his Election}

\subsection{The President's Conduct Concerning the Investigation of Michael Flynn}

\subsubsection{Incoming National Security Advisor Flynn Discusses Sanctions on Russia with Russian Ambassador Sergey Kislyak}

\subsubsection{President-Elect Trump is Briefed on the Intelligence Community's Assessment of Russian Interference in the Election and Congress Opens Election-Interference Investigations}

\subsubsection{Flynn Makes False Statements About his Communications with Kislyak to Incoming Administration Officials, the Media, and the FBI}

\subsubsection{DOJ Officials Notify the White House of Their Concerns About Flynn}

\subsubsection{McGahn has a Follow-Up Meeting About Flynn with Yates; President Trump has Dinner with FBI Director Comey}

\subsubsection{Flynn's Resignation}

\subsubsection{The President Discusses Flynn with FBI Director Comey}

\subsubsection{The Media Raises Questions About the President's Delay in Terminating Flynn}

\subsubsection{The President Attempts to Have K.T. McFarland Create a Witness Statement Denying that he Directed Flynn's Discussions with Kislyak}

\subsection{The President's Reaction to Public Confirmation of the FBI's Russia Investigation}

\subsubsection{Attorney General Sessions Recuses From the Russia Investigation}

\subsubsection{FBI Director Comey Publicly Confirms the Existence of the Russia Investigation in Testimony Before HPSCI}

\subsubsection{The President Asks Intelligence Community Leaders to Make Public Statements that he had No Connection to Russia}

\subsubsection{The President Asks Comey to “Lift the Cloud” Created by the Russia Investigation}

\subsection{Events Leading Up To and Surrounding the Termination of FBI Director Comey}

\subsubsection{Comey Testifies Before the Senate Judiciary Committee and Declines to Answer Questions About Whether the President is Under Investigation}

\subsubsection{The President Makes the Decision to Terminate Comey}

\subsection{The President's Efforts to Remove the Special Counsel}

\subsubsection{The Appointment of the Special Counsel and the President's Reaction}

\subsubsection{The President Asserts that the Special Counsel has Conflicts of Interest}

\subsubsection{The Press Reports that the President is Being Investigated for Obstruction of Justice and the President Directs the White House Counsel to Have the Special Counsel Removed}

\subsection{The President's Efforts to Curtail the Special Counsel Investigation}

\subsubsection{The President Asks Corey Lewandowski to Deliver a Message to Sessions to Curtail the Special Counsel Investigation}

\subsubsection{The President Follows Up with Lewandowski}

\subsubsection{The President Publicly Criticizes Sessions in a New York Times Interview}

\subsubsection{The President Orders Priebus to Demand Sessions's Resignation}

\subsection{The President's Efforts to Prevent Disclosure of Emails About the June 9, 2016 Meeting Between Russians and Senior Campaign Officials}

\begin{center}
\textbf{Overview}
\end{center}

By June 2017, the President became aware of emails setting up the June 9, 2016 meeting between senior campaign officials and Russians who offered derogatory information on Hillary Clinton as “part of Russia and its government's support for Mr. Trump.”
On multiple occasions in late June and early July 2017, the President directed aides not to publicly disclose the emails, and he then dictated a statement about the meeting to be issued by Donald Trump Jr. describing the meeting as about adoption.

\subsubsection{The President Learns About the Existence of Emails Concerning the June 9; 2016 Trump Tower Meeting}

In mid-June 2017—the same week that the President first asked Lewandowski to pass a message to Sessions—senior Administration officials became aware of emails exchanged during the campaign arranging a meeting between Donald Trump Jr., Paul Manafort, Jared Kushner, and a Russian attorney.“
As described in Volume I, Section IV.A.5, supra, the emails stated that the “Crown [P]rosecutor of Russia” had offered “to provide the Trump campaign with some official documents and information that would incriminate Hillary and her dealings with Russia”as part of“Russiaanditsgovernment'ssupportforMr.Trump.”°*
Trump Jr.responded,“[I]fit'swhat you say I love it,”°® and he, Kushner, and Manafort met with the Russian attorney and several other Russian individuals at Trump Tower on June 9, 2016.%
At the meeting, the Russian attorney claimed that funds derived from illegal activities in Russia were provided to Hillary Clinton and other Democrats, and the Russian attorney then spoke about the Magnitsky Act, a 2012 U.S. statute that imposed financial and travel sanctions on Russian officials and that had resulted in a retaliatory ban in Russia on U.S. adoptions of Russian children.°

According to written answers submitted by the President in response to questions from this Office, the President had no recollection of learning of the meeting or the emails setting it up at the time the meeting occurred or at any other time before the election.°

The Trump Campaign had previously received a document request from SSCTthat called for the production of various information, including, “[a] list and a description of all meetings” between any “individual affiliated with the Trump campaign” and “any individual formally or informally affiliated with the Russian government or Russian business interests which took place between June 16, 2015, and 12 pm on January 20, 2017,” and associated records.%
Trump Organization attorneys became aware of the June 9 meeting nolaterthan the first week of June 2017, when they began interviewing the meeting participants, and the Trump Organization attorneys provided the emails setting up the meeting to the President's personal counsel.°”
Mark Corallo, who had been hired as a spokesman for the President's personal legal team, recalled that he learned about the June 9 meeting around June 21 or 22, 2017.97!
Priebus recalled learning about the June 9 meeting from Fox News host Sean Hannity in late June 2017.5? Priebus notified one ofthePresident'spersonalattorneys, whotoldPriebushewasalreadyworkingonit.5
By late June, several advisors recalled receiving media inquiries that could relate to the June 9 meeting. %674

\subsubsection{The President Directs Communications Staff Not to Publicly Disclose Information About the June 9 Meeting}

Communications advisors Hope Hicks and Josh Raffel recalled discussing with Jared Kushner and Ivanka Trump that the emails were damaging and would inevitably be leaked.®”
Hicks and Raffel advised that the best strategy was to proactively release the emails to the press.”
On or about June 22, 2017, Hicks attended a meeting in the White House residence with the President, Kushner, and Ivanka Trump.°””
According to Hicks, Kushner said that he wantedtofill thePresidentinonsomethingthathadbeendiscoveredinthedocumentshewastoprovidetothe congressional committees involving a meeting with him, Manafort, and Trump Jr.978
Kushner brought a folder of documents to the meeting and tried to show them to the President, but the President stopped Kushner and said he did not want to know about it, shutting the conversation down.”

On June 28, 2017, Hicks viewed the emails at Kushner's attorney's office.*°
She recalled beingshockedbytheemailsbecausetheylooked“really bad.”°*!
Thenextday, Hicks spoke privately with the President to mention her concern about the emails, which she understood were soon going to be shared with Congress.”
The President seemed upset because too many people knewabouttheemailsandhetoldHicksthatjustonelawyershoulddealwiththematter.
The President indicated that he did not think the emails would leak, but said they would leak if everyone had access to them.°**

Later that day, Hicks, Kushner, and Ivanka Trump wenttogetherto talk to the President.*°
Hicks recalled that Kushner told the President the June 9 meeting was not a big deal and was about Russian adoption, but that emails existed setting up the meeting.®**
Hicks said she wanted to get in front of the story and have Trump Jr. release the emails as part of an interview with “softball questions.”®*'
ThePresidentsaidhedidnotwanttoknowaboutitandtheyshouldnotgotothe press.°*8 Hicks warned the President that the emails were “really bad” and the story would be “massive” when it broke, but the President was insistent that he did not want to talk about it and said he did not want details.'
Hicks recalled that the President asked Kushner when his document production was due.®° Kushner responded that it would be a couple of weeks and the President said, “then leave it alone.”©?!
Hicks also recalled that the President said Kushner's attorney should give the emails to whomever he needed to give them to, but the President did not think they would be leaked to the press.
Raffel later heard from Hicks that the President had directed the group not to be proactive in disclosing the emails because the President believed they would not leak.°TM

\subsubsection{The President Directs Trump Jr.'s Response to Press Inquiries About the June 9 Meeting}

The following week, the President departed on an overseas trip for the G20 summit in Hamburg, Germany, accompanied by Hicks, Raffel, Kushner, and Ivanka Trump, among others.°*
On July 7, 2017, while the President was overseas, Hicks and Raffel learned that the New York Times was working Ina story about the June 9 meeting.”°
The next day, Hicks told the President about the story and he directed her not to comment.®
Hicks thought the President's reaction was oddbecauseheusuallyconsiderednotrespondingtothepresstobetheultimatesin.” Later that day, Hicks and the President again spoke about the story.*”*
Hicks recalled that the President asked her what the meeting had been about, and she said that she had been told the meeting was about Russian adoption.® ThePresidentresponded,“thenjustsaythat.”7°°

Ontheflight home from the G20 on July 8, 2017, Hicks obtained a draft statement about themeetingtobereleasedbyTrumpJr.andbroughtittothePresident.' Thedraftstatement began with a reference to the information that was offered by the Russians in setting up the meeting: “I was asked to have a meeting by an acquaintance I knew from the 2013 Miss Universe pageant with an individual who I was told might have information helpful to the campaign.””” Hicks again wanted to disclose the entire story, but the President directed that the statement not be
issued because it said too much.”°? The President told Hicks to say only that Trump Jr. took a brief meeting and it was about Russian adoption.”TM
After speaking with the President, Hicks texted Trump Jr. a revised statement on the June 9 meeting that read:

It was a short meeting. I asked Jared and Paul to stop by. We discussed a program about the adoption of Russian children that was active and popular with American families years ago and was since ended by the Russian government, but it was not a campaign issue at that time and there was no follow up.”

Hicks's text concluded, “Are you ok with this? Attributed to you.””°
Trump Jr. responded by text message that he wanted to add the word “primarily” before “discussed” so that the statement would read,“WeprimarilydiscussedaprogramabouttheadoptionofRussianchildren.”?°?
Trump Jr. texted that he wanted the change because “[t]hey started with some Hillary thing which wasbs and some other nonsense which we shot down fast.””°8 Hicks texted back,“I think that's right too butbossmanworrieditinvitesalotofquestions[.]
[U]Itimately[d]efertoyouand[your attorney] on that word Be I know it's important and think the mention of a campaign issue adds something toitincasewehavetogofurther.”””
Trump Jr.responded,“IfIdon'thaveitinthereitappears asthoughI'mlyinglaterwhentheyinevitablyleaksomething.””!?
TrumpJr.'sstatement—adding the word “primarily” and making other minor additions—was then provided to the New York Times.”!!
The full statement provided to the Times stated:

It was a short introductory meeting.
I asked Jared and Paul to stop by.
We primarily discussed a program about the adoption of Russian children that was active and popular with American families years ago and was since ended by the Russian government, but it was not a campaign issue at the time and there was no follow up.
I was asked to attend the meeting by an acquaintance, but was not told the name of the person I would be meeting with beforehand.”!

ThestatementdidnotmentiontheofferofderogatoryinformationaboutClintonoranydiscussion of the Magnitsky Act or U.S. sanctions, which were the principal subjects of the meeting, as described in Volume I, Section IV.A.5, supra.

A short while later, while still on Air Force One, Hicks learned that Priebus knew about the emails, which further convincedherthat additional information about the June 9 meeting would leakandtheWhiteHouseshouldbeproactiveandgetinfrontofthestory.”'3
Hicksrecalledagain going to the President to urge him that they should be fully transparent about the June 9 meeting, but he again said no, telling Hicks, “You've given a statement.
We're done.”7"4

Later on the flight home, Hicks went to the President's cabin, where the President was on the phone with one of his personal attorneys.'!>
At one point the President handed the phone to Hicks, andtheattorneytoldHicksthathehadbeenworkingwithCircaNewsonaseparatestory, and that she should not talk to the New York Times.7!®

\subsubsection{The Media Reports on the June 9, 2016 Meeting}

Before the President's flight home from the G20 landed, the New York Times published its story about the June 9, 2016 meeting.'!”
In addition to the statement from Trump Jr., the Times story also quoted a statement from Corallo on behalf of the President's legal team suggesting that the meeting might have been a setup by individuals working with the firm that produced the Steele reporting.”'®
Corallo also worked with Circa News on a story published an hour later that questioned whether Democratic operatives had arranged the June 9 meeting to create the appearance of improper connections between Russia and Trump family members.'!?
Hicks was upset about Corallo's public statement and called him that evening to say the President had not approved the statement.'”°

The next day, July 9, 2017, Hicks and the President called Corallo together and the PresidentcriticizedCoralloforthestatementhehadreleased.””!
CorallotoldthePresidentthe statement had been authorized and further observed that Trump Jr.'s statement was inaccurate and that a document existed that would contradict it.”?
Corallo said that he purposely used the term “document”to refer to the emails setting up the June 9 meeting because he did not know what the President knew about the emails.”
Corallo recalled that when he referred to the “document” on the call with the President, Hicks responded that only a few people had access to it and said “it will never get out.”'**
Corallo took contemporaneous notes of the call that say: “Also mention existenceofdoc. Hope says‘onlyafewpeoplehaveit. Itwillnevergetout.””””>
Hickslatertold investigatorsthatshehadnomemoryofmakingthatcommentandhadalwaysbelievedtheemails would eventually be leaked, but she might have been channeling the President on the phone call because it was clear to her throughout her conversations with the President that he did not think the emails would leak.”

On July 11, 2017, Trump Jr. posted redacted images of the emails setting up the June 9 meeting on Twitter; the New York Times reported that he did so “[a]fter being told that The Times was about to publish the content of the emails.”””
Later that day, the media reported that the President had been personally involved in preparing Trump Jr.'s initial statement to the New York Times that had claimed the meeting “primarily” concerned “a program about the adoption of Russian children.”'”8
Over the next several days, the President's personal counsel repeatedly and inaccurately denied that the President played any role in drafting Trump Jr.'s statement.”°
After consulting with the President on the issue, White House Press Secretary Sarah Sanders told the media that the President “certainly didn't dictate” the statement, but that “he weighed in, offered suggestions like any father would do.””°
Several months later, the President's personal counsel stated in a private communication to the Special Counsel's Office that “the President dictated a short but accurate response to the New York Times article on behalf of his son, Donald Trump,
Jr.”73!
ThePresidentlatertoldthepressthatitwas“irrelevant”whetherhedictatedthestatement
and said, “It's a statement to the New York Times. . . . judges.””°?
That's not a statement to a high tribunal of

June 9 meeting.
On July 12, 2017, theSpecialCounsel's
related to the June 9 meeting and those who attended the
Trump Jr.

most politicians — I was just with a lot of people, they said... , meeting like that?”””*
‘Who wouldn't have taken a
On July 19, 2017, the President had his follow-up meeting with Lewandowski and then met with reporters for the New York Times.
In addition to criticizing Sessions in his Times interview, the President addressed the June 9, 2016 meeting and said he “didn't know anything about the meeting”at the time.”
The President added, “As I've said—most other people, you know, when they call up and say, ‘By the way, we have information on your opponent,' T think

\begin{center}
\textbf{Analysis}
\end{center}

In analyzing the President's actions regarding the disclosure of information about the June 9 meeting, the following evidence is relevant to the elements of obstruction of justice:

Obstructive act.
On at least three occasions between June 29, 2017, and July 9, 2017, the President directed Hicks and others not to publicly disclose information about the June 9, 2016 meeting between senior campaign officials and a Russian attorney.
On June 29, Hicks warned the President that the emails setting up the June 9 meeting were“really bad” and the story would be “massive” when it broke, but the President told her and Kushner to “leave it alone.”
Early on July 8, after Hicks told the President the New York Times was working on a story about the June 9 meeting, the President directed her not to comment, even though Hicks said that the Presidentusuallyconsiderednotrespondingtothepresstobetheultimatesin.
Laterthatday, the President rejected Trump Jr.'s draft statement that would have acknowledged that the meeting was with “an individual who I was told might have information helpful to the campaign.” The President then dictated a statement to Hicks that said the meeting was about Russian adoption (which the President had twice been told was discussed at the meeting).
The statement dictated by the President did not mention the offer of derogatory information about Clinton.

EachoftheseeffortsbythePresidentinvolvedhiscommunicationsteamandwasdirected at the press.
They would amount to obstructive acts only if the President, by taking these actions, sought to withhold information from or mislead congressional investigators or the Special Counsel.
On May 17, 2017, the President's campaign received a document request from SSCI that clearly covered the June 9 meeting and underlying emails, and those documents also plainly would have beenrelevantto the Special Counsel's investigation.

But the evidence does not establish that the President took steps to prevent the emails or other information about the June 9 meeting from being provided to Congress or the Special Counsel.
The series of discussions in which the President sought to limit access to the emails and prevent their public release occurred in the context of developing a press strategy.
The only evidence we have of the President discussing the production of documents to Congress or the Special Counsel is the conversation on June 29, 2017, when Hicks recalled the President acknowledging that Kushner's attorney should provide emails related to the June 9 meeting to whomeverheneededtogivethemto.
WedonothaveevidenceofwhatthePresidentdiscussed with his own lawyers at that time.

Nexus to an official proceeding.
As described above, by the time of the President's attempts to prevent the public release of the emails regarding the June 9 meeting, the existence of a grand jury investigation supervised by the Special Counsel was public knowledge, and the President had been told that the emails were responsive to congressional inquiries.
To satisfy the nexus requirement, however, it would be necessary to show that preventing the release of the emails to the public would have the natural and probable effect of impeding the grand jury proceeding or congressional inquiries.
As noted above, the evidence does not establish that the President sought to prevent disclosure of the emails in those official proceedings.

Intent.
The evidence establishes the President's substantial involvement in the communications strategy related to information about his campaign's connections to Russia and his desire to minimize public disclosures about those connections.
The President became aware of the emails no later than June 29, 2017, when he discussed them with Hicks and Kushner, and he could have been aware of them as early as June 2, 2017, when lawyers for the Trump Organization began interviewing witnesses who participated in the June 9 meeting.
The President thereafterrepeatedlyrejectedtheadviceofHicksandotherstafferstopubliclyreleaseinformation about the June 9 meeting.
The President expressed concern that multiple people had access to the emails and instructed Hicks that only one lawyer should deal with the matter.
And the President dictatedastatementtobereleasedbyTrumpJr.inresponsetothefirstpressaccountsoftheJune 9 meeting that said the meeting was about adoption.

But as described above, the evidence does not establish that the President intended to prevent the Special Counsel's Office or Congress from obtaining the emails setting up the June 9 meeting or other information about that meeting.
The statement recorded by Corallo—that the emails “will never get out”—can be explained as reflecting a belief that the emails would not be made public if the President's press strategy were followed, even if the emails were provided to Congress and the Special Counsel.

\subsection{The President's Further Efforts to Have the Attorney General Take Over the Investigation}

Overview

From summer 2017 through 2018, the President attempted to have Attorney General Sessions reverse his recusal, take control of the Special Counsel's investigation, and order an investigation of Hillary Clinton.

Evidence

\subsubsection{The President Again Seeks to Have Sessions Reverse his Recusal}

After returning Sessions's resignation letter at the end of May 2017, but before the President's July 19, 2017 New York Times interview in whichhepublicly criticized Sessions for recusing from the Russia investigation, the President took additional steps to have Sessions reverse his recusal.
In particular, atsomepointaftertheMay17, 2017appointmentoftheSpecialCounsel, Sessions recalled, the President called him at home and asked if Sessions would “unrecuse” himself.”2°
According to Sessions, the President asked him to reverse his recusal so that Sessions could direct the Department of Justice to investigate and prosecute Hillary Clinton, and the “gist” of the conversation was that the President wanted Sessions to unrecuse from “allofit,” including the Special Counsel's Russia investigation.'
Sessions listened but did not respond, and he did not reverse his recusal or order an investigation of Clinton.”*

In early July 2017, the President asked Staff Secretary Rob Porter what he thought of Associate Attorney General Rachel Brand.”
Porter recalled that the President asked him if Brand was good, tough, and “on the team.””°
The President also asked if Porter thought Brand was interested in being responsible for the Special Counsel's investigation and whether she would want to be Attorney General one day.”!
Because Porter knew Brand, the President asked him to sound her out about taking responsibility for the investigation and being Attorney General.”?
Contemporaneousnotestaken by Porter show that the President told Porter to “Keep in touch with your friend,” in reference to Brand.”
Later, the President asked Porter a few times in passing whether he had spoken to Brand, but Porter did not reach out to her because he was uncomfortable with the task.”“4
In asking him to reach out to Brand, Porter understood the President to want to find someone to end the Russia investigation or fire the Special Counsel, although the President never said so explicitly.°
Porter did not contact Brand because he was sensitive to the implicationsofthatactionanddidnotwanttobeinvolvedinachainofeventsassociatedwithan effort to end the investigation or fire the Special Counsel.”4°

McGahnrecalledthatduringthesummerof2017, heandthePresidentdiscussedthefact thatifSessionswerenolongerinhispositiontheSpecialCounselwouldreportdirectlytoanon- recused Attorney General.'
McGahn told the President that things might not change much under a new Attorney General.”48
McGahn also recalled that in or around July 2017, the President frequently brought up his displeasure with Sessions.”
Hicks recalled that the President viewed Sessions'srecusalfromtheRussiainvestigationasanactofdisloyalty.'°°
Inadditiontocriticizing Sessions's recusal, the President raised other concerns about Sessions and his job performance with McGahn and Hicks.'*!

\subsubsection{Additional Efforts to Have Sessions Unrecuse or Direct Investigations Covered by his Recusal}

Later in 2017, the President continued to urge Sessions to reverse his recusal from campaign-related investigations and considered replacing Sessions with an Attorney General who would not be recused.

On October 16, 2017, the President met privately with Sessions and said that the Department of Justice was not investigating individuals and events that the President thought the Department should be investigating.””?
According to contemporaneous notes taken by Porter, who was at the meeting, the President mentioned Clinton's emails and said, “Don't havetotell us, just take [a] look.””°?
Sessions did not offer any assurances or promises to the President that the Department of Justice would comply with that request.'**
Two days later, on October 18, 2017, the President tweeted, “Wow, FBI confirms report that James Comey drafted letter exonerating Crooked Hillary Clinton long before investigation was complete.
Many people not interviewed, including Clinton herself.
Comey stated under oath that he didn't do this-obviously a fix? Where is Justice Dept?””*° On October 29, 2017, the President tweeted that there was “ANGER & UNITY”over a “lack of investigation” of Clinton and “the Comey fix,” and concluded: “DO SOMETHING!”

On December6, 2017, five days after Flynn pleaded guilty to lying about his contacts with the Russian government, the President asked to speak with Sessions in the Oval Office at the end of a cabinet meeting.”"'
During that Oval Office meeting, which Porter attended, the President again suggested that Sessions could “unrecuse,” which Porter linked to taking back supervision of the Russia investigation and directing an investigation of Hillary Clinton.”*
According to contemporaneousnotestaken by Porter, the President said, “I don't know if you could un-recuse yourself.
You'd be a hero.
Not telling you to do anything.
Dershowitz says POTUS can get involved. CanorderAGtoinvestigate.
Idon'twanttogetinvolved.
I'mnotgoingtogetinvolved.
I'm not going to do anything or direct you to do anything.
I just want to be treated fairly.””*
According to Porter's notes, Sessions responded, “We are taking steps; whole new leadership team. Professionals; will operate according to the law.””°°
Sessions also said, “I never saw anything that was improper,” which Porter thought was noteworthy because it did not fit with the previous discussion about Clinton.”°!
Porter understood Sessions to be reassuring the President that he was on the President's team.'

At the end of December, the President told the New York Times it was “too bad”that SessionshadrecusedhimselffromtheRussiainvestigation.'
WhenaskedwhetherHolderhad been a more loyal Attorney General to President Obama than Sessions was to him, the President said, “I don't want to get into loyalty, but I will tell you that, I will say this: Holder protected President Obama.
Totally protected him.
When youlookatthe things that they did, and Holder protected the president.
And I have great respect for that, 'll be honest.””** Later in January, the PresidentbroughtuptheideaofreplacingSessionsandtoldPorterthathewantedto“clean house” at the DepartmentofJustice.”
In a meeting in the White House residence that Porter attended on January 27, 2018, Porter recalled that the President talked about the great attorney she had in the past with successful win records, such as Roy Cohn and Jay Goldberg, and said that one of his biggest failings as President was that he had not surrounded himself with good attorneys, citing Sessions as an example.”°
The President raised Sessions's recusal and brought up and criticized the Special Counsel's investigation.”

Over the next several months, the President continued to criticize Sessions in tweets and media interviews and on several occasions appeared to publicly encourage him to take action in the Russia investigation despite his recusal.”*
On June 5, 2018, for example, the President tweeted, “The Russian Witch Hunt Hoax continues, all because Jeff Sessions didn't tell me he was going to recuse himself. ... I would have quickly picked someone else.
So much time and money wasted, so many lives ruined . . . and Sessions knew better than most that there was No Collusion!””
On August 1, 2018, the President tweeted that “Attorney General Jeff Sessions should stop this Rigged Witch Hunt right now.”'”°
On August 23, 2018, the President publicly criticized Sessions in a press interview and suggested that prosecutions at the Department of Justice were politically motivated because Paul Manafort had been prosecuted but Democrats had not.'””! ThePresidentsaid,“IputinanAttorneyGeneralthatnevertookcontroloftheJustice Department, Jeff Sessions.”””?
That day, Sessionsissueda pressstatementthatsaid,“Itookcontrol oftheDepartmentofJusticethedayIwasswornin....
WhileIamAttorneyGeneral, the actions of the DepartmentofJustice will not be improperly influencedbypolitical considerations.”””3
The next day, the President tweeted a response: “‘Department of Justice will not be improperly influencedbypolitical considerations.'
Jeff, this is GREAT, what everyone wants, so look into all of the corruption on the ‘other side' including deleted Emails, Comey lies & leaks, Mueller conflicts, McCabe, Strzok, Page, Ohr, FISA abuse, Christopher Steele & his phony and corrupt Dossier, the Clinton Foundation, illegal surveillance of Trump campaign, Russian collusion by Dems— and so much more.
Open up the papers & documents without redaction? ComeonJeff, you can do it, the country is waiting!”””

On November7, 2018, the day after the midterm elections, the President replaced Sessions with Sessions's chiefofstaffas Acting Attorney General.””

Analysis

In analyzing the President's efforts to have Sessions unrecuse himself and regain control of the Russia investigation, the following considerations and evidence are relevant to the elements of obstruction of justice:

Obstructive act.
 To determine if the President's efforts to have the Attorney General unrecuse could qualify as an obstructive act, it would be necessary to assess evidence on whether those actions would naturally impede the Russia investigation.
That inquiry would take into account the supervisory role that the Attorney General, if unrecused, would play in the Russia investigation.
It also would have to take into account that the Attorney General's recusal covered other campaign-related matters.
The inquiry would not turn on what Attorney General Sessions would actually do if unrecused, but on whether the efforts to reverse his recusal would naturally havehadthe effect of impeding the Russia investigation.

On multiple occasions in 2017, the President spoke with Sessions about reversing his recusal so that he could take over the Russia investigation and begin an investigation and prosecution of Hillary Clinton.
For example, in early summer 2017, Sessions recalled the President asking him to unrecuse, but Sessions did not take it as a directive.
When the President raised the issue again in December 2017, the President said, as recorded by Porter, “Not telling you to do anything. ...
I'm not going to get involved.
I'm not going to do anything or direct you todoanything.
Ijustwanttobetreatedfairly.”
ThedurationofthePresident'sefforts—which spanned from March 2017 to August 2018—andthefact that the President repeatedly criticized Sessions in public and in private for failing to tell the President that he would have to recuse is relevant to assessing whether the President's efforts to have Sessions unrecuse could qualify as obstructive acts.

Nexus to an official proceeding.
As described above, by mid-June 2017, the existence ofagrandjuryinvestigationsupervisedbytheSpecialCounselwaspublicknowledge.
In addition, in July 2017, a different grand jury supervised by the Special Counsel was empaneled in the District of Columbia, and the press reported on the existence of this grand jury in early August 2017.7
Whether the conduct towards the Attorney General would have a foreseeable impact on those proceedings turns on much of the same evidence discussed above with respect to the obstructive-act element.

Intent.
There is evidence that at least one purpose of the President's conduct toward Sessions was to have Sessions assume control over the Russia investigation and supervise it in a waythatwouldrestrictitsscope.
Bythesummerof2017, thePresidentwasawarethattheSpecial Counsel was investigating him personally for obstruction of justice.
And in the wake of the disclosures of emails about the June 9 meeting between Russians and senior members of the campaign, see Volume II, Section II.G, supra, it was evident that the investigation into the campaign now included the President's son, son-in-law, and former campaign manager.
The President had previously and unsuccessfully sought to have Sessions publicly announce that the Special Counsel investigation would be confined to future election interference.
Yet Sessions remained recused.
In December 2017, shortly after Flynn pleaded guilty, the President spoke to Sessions in the Oval Office with only Porter present and told Sessions that he would be a hero if he unrecused.
Porter linked that request to the President's desire that Sessions take back supervision of the Russia investigation and direct an investigation of Hillary Clinton.
The President said in that meeting that he “just want[ed] to be treated fairly,” which could reflect his perception that it was unfair that he was being investigated while Hillary Clinton was not.
But a principal effect of that act would be to restore supervision of the Russia investigation to the
Attorney General—aposition that the President frequently suggested should be occupied by someone like Eric Holder and Bobby Kennedy, who the President described as protecting their presidents.
A reasonable inference from those statements and the President's actions is that the President believed that an unrecused Attorney General would play a protective role and could shield the President from the ongoing Russia investigation.

\subsection{The President Orders McGahn to Deny that the President Tried to Fire the Special Counsel}

Overview

In late January 2018, the media reported that in June 2017 the President had ordered McGahn to have the Special Counsel fired based on purported conflicts of interest but McGahn had refused, saying he would quit instead.
After the story broke, the President, through his personal counsel and two aides, sought to have McGahndenythat he had been directed to remove the Special Counsel.
Each time he was approached, McGahnrespondedthat he would not refute the press accounts because they were accurate in reporting on the President's effort to have the Special Counsel removed.
The President later personally met with McGahn in the Oval Office withonlytheChiefofStaffpresentandtriedtogetMcGahntosaythatthePresidentneverordered him to fire the Special Counsel.
McGahn refused and insisted his memory of the President's direction to remove the Special Counsel was accurate.
In that same meeting, the President challenged McGahn for taking notes of his discussions with the President and asked why he had told Special Counsel investigators that he had been directed to have the Special Counsel removed.

Evidence

\subsubsection{The Press Reports that the President Tried to Fire the Special Counsel}

On January 25, 2018, the New York Times reported that in June 2017, the President had ordered McGahn to have the DepartmentofJustice fire the Special Counsel.”””
According to the article, “[a]mid the first wave of news media reports that Mr. Mueller was examining a possible obstruction case, the president began to argue that Mr. Mueller had three conflicts of interest that disqualified him from overseeing the investigation.”””*
The article further reported that “[a]fter
receiving the president's order to fire Mr. Mueller, the White House counsel .. .
Justice Department to dismiss the special counsel, saying he would quit instead.”””?
The article stated that the president “ultimately backed down after the White House counsel threatened to resign rather than carry out the directive.””*°
After the article was published, the President
refused to ask the
dismissed the story when asked about it by reporters, saying, “Fake news, folks.
Fake news.
A typical New York Times fake story.””*!

The next day, the Washington Post reported on the same event but added that McGahn had not told the President directly that he intended to resign rather than carry out the directive to have the Special Counsel terminated.”*? In that respect, the Post story clarified the Times story, which could be read to suggest that McGahn had told the President of his intention to quit, causing the President to back down from the order to have the Special Counsel fired.”

\subsubsection{The President Seeks to Have McGahn Dispute the Press Reports}

On January 26, 2018, the President's personal counsel called McGahn'sattorney and said that the President wanted McGahn to put out a statement denying that he had been askedtofire the Special Counsel and that he had threatened to quit in protest.”**
McGahn's attorney spoke with McGahn about that request and then called the President's personal counsel to relay that McGahn wouldnotmakea statement.'”*>
McGahn'sattorneyinformedthePresident'spersonalcounselthat the Times story was accurate in reporting that the President wanted the Special Counsel removed.”
Accordingly, McGahn's attorney said, although the article was inaccurate in some other respects, McGahn could not comply with the President's request to dispute the story.'*” HicksrecalledrelayingtothePresidentthatoneofhisattorneyshadspokentoMcGahn'sattorney about the issue.”**

Also on January 26, 2017, Hicks recalled that the President asked Sanders to contact McGahnaboutthestory.”° McGahntoldSanderstherewasnoneedtorespondandindicatedthat someofthe article was accurate.”
Consistent with that position, McGahn did not correct the Times story.

On February 4, 2018, Priebus appeared on Meet the Press and said he had not heard the PresidentsaythathewantedtheSpecialCounselfired.”' AfterPriebus'sappearance, the PresidentcalledPriebusandsaidhedidagreatjobonMeetthePress.””?
ThePresidentalsotold PriebusthatthePresidenthad“neversaidanyofthosethingsabout”theSpecialCounsel.'

The next day, on February 5, 2018, the President complained about the Times article to Porter.”
The President told Porter that the article was “bullshit” and he had not sought to terminate the Special Counsel.”
The President said that McGahn leaked to the media to make himselflookgood.”°
ThePresidentthendirectedPortertotellMcGahntocreatearecordtomake clear that the President never directed McGahn to fire the Special Counsel.'
Porter thought the matter should be handled by the White House communications office, but the President said he wantedMcGahntowritealettertothefile“forourrecords”andwantedsomethingbeyondapress statement to demonstrate that the reporting was inaccurate.'”°
The President referred to McGahn as a “lying bastard” and said that he wanted a record from him.””
Porter recalled the President sayingsomethingtotheeffectof,“Ifhedoesn'twritealetter, thenmaybeI'llhavetogetridof him.”

Laterthatday, PorterspoketoMcGahntodeliverthePresident'smessage.®”!
Porter told McGahn that he had to write a letter to dispute that he was ever ordered to terminate the Special Counsel.
McGahn shrugged off the request, explaining that the media reports were true.”
McGahn told Porter that the President had been insistent on firing the Special Counsel and that McGahnhadplannedto resign rather than carry out the order, although he had not personally told thePresidentheintendedtoquit.®*
PortertoldMcGahnthatthePresidentsuggestedthatMcGahn wouldbefiredifhedidnotwritetheletter.8°>
McGahndismissedthethreat, sayingthattheoptics would be terrible if the President followed through with firing him on that basis.°°°
McGahn said he would not write the letter the President had requested.8”'
Porter said that to his knowledge the
issueofMcGahn'sletternevercameupwiththePresidentagain, butPorterdidrecalltellingKelly about his conversation with McGahn.®®

The next day, on February 6, 2018, Kelly scheduled time for McGahn to meet with him and the President in the Oval Office to discuss the Times article.8°'
The morning of the meeting, the President's personal counsel called McGahn's attorney and said that the President was going to be speaking with McGahn and McGahn could not resign no matter what happened in the meeting.*!°

The President began the Oval Office meeting by telling McGahn that the New York Times story did not “look good” and McGahn needed to correct it.!'
McGahn recalled the President said, “I never said to fire Mueller.
I never said ‘fire.'
This story doesn't look good.
You need to correct this.
You're the White House counsel.”*!?

In response, McGahn acknowledged that he had not told the President directly that he planned to resign, but said that the story was otherwise accurate.*'?
The President asked McGahn, “Did I say the word ‘fire'?”8'*
McGahn responded, “What you said is, ‘Call Rod [Rosenstein], tellRodthatMuellerhasconflictsandcan'tbetheSpecialCounsel.'”*'®
ThePresidentresponded, “I never said that.'8'®
The President said he merely wanted McGahntoraise the conflicts issue with Rosenstein andleaveit to him to decide what to do.*!7
McGahntoldthe President he did not understand the conversation that way and instead had heard, “Call Rod. There are conflicts. Mueller has to go.”8!8
The President asked McGahn whether he would “do a correction,” and McGahn said no.®!?
McGahn thought the President was testing his mettle to see how committed McGahnwasto what happened.*°
Kelly described the meeting as“a little tense.”*?!

ThePresidentalsoaskedMcGahninthemeetingwhyhehadtoldSpecialCounsel'sOffice investigators that the President had told him to have the Special Counsel removed.7?
McGahn responded that he had to and that his conversations with the President were not protected by attorney-client privilege.5??
ThePresidentthenasked,“Whataboutthesenotes? Whydoyoutake notes? Lawyers don't take notes.
I never had a lawyer who took notes.”**
McGahn responded that he keeps notes becauseheis a “real lawyer” and explained that notes create a record and are notabadthing.2
ThePresidentsaid,“I'vehadlotofgreatlawyers, likeRoyCohn. Hedidnot take notes.'®”

After the Oval Office meeting concluded, Kelly recalled McGahn telling him that McGahn and the President “did have that conversation” about removing the Special Counsel.”
McGahn recalled that Kelly said that he had pointed out to the President after the Oval Office that McGahn had not backed down and would not budge.®”8
Following the Oval Office meeting, the President's personal counsel called McGahn's counsel and relayed that the President was “fine” with McGahn.*”?

Analysis

In analyzing the President's efforts to have McGahn deny that he had been ordered to have theSpecialCounselremoved, thefollowingevidenceisrelevanttotheelementsofobstructionof
justice:

a. Obstructive act. The President's repeated efforts to get McGahn to create a record denying that the President had directed him to remove the Special Counsel would qualify as an obstructive act if it had the natural tendency to constrain McGahn from testifying truthfully or to undermine his credibility as a potential witness if he testified consistently with his memory, rather than with what the record said.

There is some evidence that at the time the New York Times and Washington Post stories werepublishedin late January 2018, the President believed the stories were wrong and that he had never told McGahn to have Rosenstein remove the Special Counsel.
The President correctly understood that McGahnhadnottold the President directly that he planned to resign.
In addition, the President told Priebus and Porter that he had not sought to terminate the Special Counsel, and in the Oval Office meeting with McGahn, the President said, “I never said to fire Mueller.
I never said ‘fire.””
That evidence could indicate that the President was not attempting to persuade McGahn to change his story but was instead offering his own—but different—recollection of the substanceofhisJune2017conversationswithMcGahnandMcGahn'sreactiontothem.

Other evidence cuts against that understanding of the President's conduct.
As previously described, see Volume II, Section ILE, supra, substantial evidence supports McGahn's account that the President had directed him to have the Special Counsel removed, including the timing and context of the President's directive; the manner in which McGahn reacted; and the fact that the President had been told the conflicts were insubstantial, were being considered by the Department of Justice, and should be raised with the President's personal counsel rather than brought to McGahn.
In addition, the President's subsequent denials that he had told McGahn to have the Special Counsel removed were carefully worded.
When first asked about the New York Times story, the President said, “Fake news, folks. Fake news. A typical New York Timesfakestory.” And when the President spoke with McGahn in the Oval Office, he focused on whether he had used the word“fire,” saying, “I never said to fire Mueller.
I never said ‘fire'” and “Did I say the word ‘fire'?” The President's assertion in the Oval Office meeting that he had never directed McGahn to have the Special Counsel removed thus runs counter to the evidence.

In addition, even if the President sincerely disagreed with McGahn's memory of the June 17, 2017 events, the evidence indicates that the President knew by the time of the Oval Office
meeting that McGahn's account differed and that McGahn was firm in his views.
Shortly after the story broke, the President's counsel told McGahn's counsel that the President wanted McGahn to make a statement denying he had been asked to fire the Special Counsel, but McGahn responded through his counsel that that aspect of the story was accurate and he therefore could not comply with the President's request. .
The President then directed Sanders to tell McGahn to correct the story, but McGahn told her he would not do so because the story was accurate in reporting on the President's order.
Consistent with that position, McGahnneverissued a correction.
More than a week later, the President brought up the issue again with Porter, made comments indicating the President thought McGahn had leaked the story, and directed Porter to have McGahn create a record denying that the Presidenthadtriedto fire the Special Counsel.
At that point, the President said he might “have to get rid of' McGahn ifMcGahndid not comply.
McGahn again refused and told Porter, as he had told Sanders and as his counsel had told the President's counsel, that the President had in fact ordered him to have Rosenstein remove the Special Counsel.
That evidence indicatesthatbythetimeoftheOvalOfficemeetingthePresidentwasawarethatMcGahndidnot think the story was false and did not want to issue a statement or create a written record denying facts that McGahn believed to be true.
The President nevertheless persisted and asked McGahn to repudiate facts that McGahn had repeatedly said were accurate.

b. Nexus to an official proceeding.
By January 2018, the Special Counsel's use of a grand jury had been further confirmed by the return of several indictments.
The President also was aware that the Special Counsel was investigating obstruction-related events because, among other reasons, on January 8, 2018, the Special Counsel's Office provided his counsel with a detailed list of topics for a possible interview with the President.°
The President knew that McGahn had personal knowledge of manyofthe events the Special Counsel was investigating and that McGahn had already been interviewed by Special Counsel investigators.
And in the Oval Office meeting, the President indicated he knew that McGahn had told the Special Counsel's Office about the President's effort to remove the Special Counsel.
The President challenged McGahn for disclosing that information and for taking notes that he viewed as creating unnecessary legal exposure.
That evidence indicates the President's awareness that the June 17, 2017 events were relevant to the Special Counsel's investigation and any grand jury investigation that might grow outofit.

To establish a nexus, it would be necessary to show that the President's actions would have the natural tendency to affect such a proceeding or that they would hinder, delay, or prevent the communication of information to investigators.
Because McGahn had spoken to Special Counsel investigators before January 2018, the President could not have been seeking to influence his prior statements in those interviews.
But because McGahnhadrepeatedly spoken to investigators and the obstruction inquiry was not complete, it was foreseeable that he would be interviewed again on obstruction-related topics.
If the President were focused solely on a press strategy in seeking to have McGahn refute the New York Times article, a nexus to a proceeding or to further investigative interviews would not be shown.
But the President's efforts to have McGahn write a letter“forourrecords”approximatelytendaysafterthestorieshadcomeout—wellpastthetypical time to issue a correction for a news story—indicates the President was not focused solely on a press strategy, but instead likely contemplated the ongoing investigation and any proceedings arising from it.

©; Intent. Substantial evidence indicates that in repeatedly urging McGahn to dispute that he was ordered to have the Special Counsel terminated, the President acted for the purpose of influencing McGahn's account in order to deflect or prevent further scrutiny of the President's conduct towards the investigation.

inquiry.
Several facts support that conclusion.
The President made repeated attempts to get McGahn to change his story.
As described above, by the time of the last attempt, the evidence suggests that the President had been told on multiple occasions that McGahn believed the President had ordered him to have the Special Counsel terminated.
McGahn interpreted his encounter with the President in the Oval Office as an attempt to test his mettle and see how committed he was to his memory of what had occurred.
The President had already laid the groundwork for pressing McGahn to alter his accountbytelling Porter that it might be necessary to fire McGahn if he did not deny the story, and Porter relayed that statement to McGahn.
Additional evidence of the President's intent may be gleaned from the fact that his counsel was sufficiently alarmed by the prospectofthePresident'smeetingwithMcGahnthathecalledMcGahn'scounselandsaidthat McGahn could not resign no matter what happened in the Oval Office that day.
The President's counsel was well aware of McGahn's resolve not to issue what he believed to be a false account ofeventsdespitethePresident'srequest.
Finally, asnotedabove, thePresidentbroughtupthe Special Counsel investigation in his Oval Office meeting with McGahnandcriticized him for telling this Office about the June 17, 2017 events.
The President's statements reflect his understanding—and his displeasure—that those events would be part of an obstruction-of-justice

\subsection{The President's Conduct Towards Flynn, Manafort, ████████}

Overview

In addition to the interactions with McGahn described above, the President has taken other actions directed at possible witnesses in the Special Counsel's investigation, including Flynn, Manafort, Geaandasdescribedinthenextsection, Cohen.
WhenFlynnwithdrewfromajoint defense agreement with the President, the President's personal counsel stated that Flynn's actions would be viewed as reflecting “hostility” towards the President.
During Manafort's prosecution andwhilethejurywasdeliberating, thePresidentrepeatedlystatedthatManafortwasbeing treated unfairly andmadeitknownthatManafortcouldreceiveapardon.

Evidence

\subsubsection{Conduct Directed at Michael Flynn}

As previously noted, see Volume II, Section II.B, supra, the President asked for Flynn's resignation on February 13, 2017.
Following Flynn's resignation, the President made positive public comments about Flynn, describing him as a “wonderful man,”“a fine person,” and a “very good person.”*?!
ThePresidentalso privately asked advisors to pass messages to Flynn conveying that the President still cared about him and encouraging him to stay strong.*?

InlateNovember2017, FlynnbegantocooperatewiththisOffice. On November22, 2017, FlynnwithdrewfromajointdefenseagreementhehadwiththePresident.*°? Flynn'scounseltold the President's personal counsel and counsel for the White House that Flynn could no longer have confidential communications with the White House or the President.83* Later that night, the President's personal counsel left a voicemail for Flynn's counsel that said:

I understand your situation, but let me see if I can't state it in starker terms. ...
[I]t wouldn't surprise me if you've gone on to make a deal with... the government....
[I]f ... there's information that implicates the President, then we've got a national security issue, ... so, you know, ... weneedsomekindofheadsup.
Um, justforthesakeof protecting all our interests if we can. ...
[R]emember what we'vealwayssaid about the President and his feelings toward Flynn and, that still remains ... 8°

On November23, 2017, Flynn's attorneys returned the call from the President's personal counsel to acknowledge receipt of the voicemail.5*°
Flynn's attorneys reiterated that they were no longer in a position to share information under any sort of privilege.*°'
According to Flynn's attorneys, the President's personal counsel was indignant and vocal in his disagreement.*
The President'spersonalcounselsaidthatheinterpretedwhattheysaidtohimasareflectionofFlynn's hostilitytowardsthePresidentandthatheplannedtoinformhisclientofthatinterpretation.*?
Flynn's attorneys understood that statement to be an attempt to make them reconsider their position because the President's personal counsel believed that Flynn would be disturbed to know that such a message would be conveyed to the President.*°

On December 1, 2017, Flynn pleaded guilty to making false statements pursuant to a cooperation agreement.**!
Thenextday, thePresidenttoldthepressthathewasnotconcerned about what Flynn might tell the Special Counsel.*?
In response to a question about whether the President still stood behind Flynn, the President responded, “We'll see what happens.”*
Over the next several days, the President made public statements expressing sympathy for Flynn and indicating he had not be entreated fairly.““*
On December 15, 2017, the President responded to a press inquiry about whether he was considering a pardon for Flynn by saying,“I don't wanttotall about pardons for Michael Flynn yet.
We'll see what happens.
Let's see.
I can say this: When you look at what's gone on with the FBI and with the Justice Department, people are very, very

\subsubsection{Conduct Directed at Paul Manafort}

On October27, 2017, agrandjuryintheDistrictofColumbiaindictedManafortandformer deputy campaign manager Richard Gates on multiple felony counts, and on February 22, 2018, a grand jury in the Eastern District of Virginia indicted Manafort and Gates on additional felony counts.*46 The charges in both cases alleged criminal conduct by Manafort that began as early as 2005 and continued through 2018.84”

In January 2018, Manafort told Gates that he had talked to the President's personal counsel andtheywere“goingtotakecareofus.”*4* ManaforttoldGatesitwasstupidtoplead, saying that he had been in touch with the President's personal counsel and repeating that they should “sit tight” and “we'll be taken care of.'*4? Gates asked Manafort outright if anyone mentioned pardons and Manafort said no one used that word.

As the proceedings against Manafort progressed in court, the President told Porter that he never liked Manafort and that Manafort did not know what he was doing on the campaign.®*! The President discussed with aides whether and in what way Manafort might be cooperating with the Special Counsel's investigation, and whether Manafort knew any information that would be harmful to the President.*>?

In public, the President made statements criticizing the prosecution and suggesting that Manafort was being treated unfairly. On June 15, 2018, before a scheduled court hearing that day on whether Manafort's bail should be revoked based on new charges that Manafort had tampered with witnesses while out on bail, the President told the press, “I feel badly about a lot of them because I think a lot of it is very unfair. I mean, I look at some of them where they go back 12 years. Like Manafort has nothing to do with our campaign. But I feel so—Itell you, I feel little badly about it. They went back 12 years to get things that he did 12 years ago? ... I feel badly for some people, because they've gone back 12 years to find things about somebody, and I don't think it's right.”8°? In response to a question about whether he was considering a pardon for Manafort or other individuals involved in the Special Counsel's investigation, the President said, “T don't want to talk about that. No, I don't want to talk about that. ... But look, I do want to see people treated fairly. That's what it's all about.” *°* Hours later, Manafort's bail was revoked and the President tweeted, “Wow, what a tough sentence for Paul Manafort, who has represented Ronald Reagan, Bob Dole and many other top political people and campaigns. Didn't know Manafort was the head of the Mob. What about Comey and Crooked Hillary and all the others? Very unfair!””®°>

Immediately following the revocation of Manafort's bail, the President's personal lawyer, Rudolph Giuliani, gave a series of interviews in which he raised the possibility of a pardon for Manafort. Giuliani told the New York Daily News that “[w]hen the whole thing is over, things might get cleaned up with some presidential pardons.”®°° Giuliani also said in an interview that, although the President should not pardon anyone while the Special Counsel's investigation was ongoing, “when the investigation is concluded, he's kind of on his own, right?”*? In a CNN interview two days later, Giuliani said, “T guess I should clarify this once and for all.... The president has issued no pardons in this investigation. The president is not going to issue pardons inthisinvestigation....Whenit'sover, hey, he'sthepresidentoftheUnitedStates. He retains his pardon power. Nobody is taking that away from him.”*°* Giuliani rejected the suggestion that his and the President's comments could signal to defendants that they should not cooperate in a criminal prosecution because a pardon might follow, saying the comments were “certainly not intendedthatway.”®°? Giulianisaidthecommentsonlyacknowledgedthatanindividualinvolved intheinvestigationwouldnotbe“excludedfrom[apardon],ifinfactthepresidentandhisadvisors
cometotheconclusionthatyouhavebeentreatedunfairly.”*°° Giulianiobservedthatpardons
.
were not unusual in political investigations but said, “That doesn't mean they're going to happen here. Doesn't mean that anybody should rely on it. . . . Big signal is, nobody has been pardoned yet.86!

On July 31, 2018, Manafort's criminal trial began in the Eastern District of Virginia, generating substantial news coverage.*The next day, the President tweeted, “This is a terrible situationandAttorneyGeneralJeffSessionsshouldstopthisRiggedWitchHuntrightnow, before it continues to stain our country any further. Bob Mueller is totally conflicted, and his 17 Angry Democrats that are doing his dirty work are a disgrace to USA!”*? Minutes later, the President tweeted, “Paul Manafort worked for Ronald Reagan, Bob Dole and many other highly prominent and respected political leaders. He worked for me for a very short time. Why didn't government tell me that he was under investigation. These old charges have nothing to do with Collusion—a Hoax!”8* Later in the day, the President tweeted, “Looking back on history, who was treated worse, Alfonse Capone, legendary mob boss, killer and ‘Public Enemy Number One,' or Paul Manafort, political operative & Reagan/Dole darling, now serving solitary confinement—although convicted of nothing? Where is the Russian Collusion?”*° The President's tweets about the Manafort trial were widely covered by the press.*° When asked about the President's tweets, Sanders told the press, “Certainly, the President's been clear. He thinks Paul Manafort's been treated unfairly.”®%

On August16, 2018, the Manafort case was submitted to the jury and deliberations began. At that time, Giuliani had recently suggested to reporters that the Special Counsel investigation needed to be “done in the next two or three weeks,”8°* and media stories reported that a Manafort acquittal would add to criticism that the Special Counsel investigation was not worth the time and expense, whereasa conviction could show that ending the investigation would be premature.®?

On August 17, 2018, as jury deliberations continued, the President commented on the trial from the South Lawn of the White House. In an impromptu exchange with reporters that lasted approximately five minutes, the President twice called the Special Counsel's investigation a “rigged witch hunt.”*”° When asked whether he would pardon Manafort if he was convicted, the President said, “I don't talk about that now. I don't talk about that.”®7! The President then added, without being asked a further question,“I think the whole Manafort trial is very sad when you look at what's going on there. I think it's a very sad day for our country. He worked for me for a very short period of time. But you know what, he happens to be a very good person. AndI think it's very sad what they've done to Paul Manafort.”*”? The President did not take further questions.” In response to the President's statements, Manafort's attorney said, “Mr. Manafort really appreciates the support of President Trump.”®”

On August 21, 2018, the jury found Manafort guilty on eight felony counts. Also on August 21, Michael Cohen pleaded guilty to eight offenses, including a campaign-finance violation that he said had occurred “in coordination with, and at the direction of, a candidate for federal office.”8”> The President reacted to Manafort's convictions that day by telling reporters, “Paul Manafort's a good man” and “it's a very sad thing that happened.”*”* The President described the Special Counsel's investigation as “a witch hunt that ends in disgrace.”®”” The next day, thePresidenttweeted,“IfeelverybadlyforPaulManafortandhiswonderfulfamily. ‘Justice' took a 12 year old tax case, among other things, applied tremendous pressure on him and, unlike Michael Cohen, herefusedto‘break'—makeupstoriesinordertogeta‘deal.' Suchrespectfor a brave man!”878

In a Fox News interview on August 22, 2018, the President said: “[Cohen] make a better deal when he uses me, like everybody else. And one of the reasons I respect Paul Manafort so much is he went through that trial—you know they makeupstories. People make up stories. This whole thing about flipping, they call it, I know all about flipping.”*”? The President said that flipping was “not fair” and “almost ought to be outlawed.”*®° In response to a question about whether he was considering a pardon for Manafort, the President said, “I have great respect for what he's done, in terms of what he's gone through. ... He worked for many, many people many, many years, and I would say what he did, some of the charges they threw against him, every consultant, every lobbyist in Washington probably does.”**! Giuliani told journalists that the President “really thinks Manafort has been horribly treated” and that he and the President had discussed the political fallout if the President pardoned Manafort.*** The next day, Giuliani told the Washington Post that the President had asked his lawyers for advice on the possibility of a pardon for Manafort and other aides, and had been counseled against considering a pardon until the investigation concluded.**

On September 14, 2018, Manafort pleaded guilty to charges in the District of Columbia and signed a plea agreement that required him to cooperate with investigators.“ Giuliani was reported to have publicly said that Manafort remained in a joint defense agreement with the President following Manafort's guilty plea and agreement to cooperate, and that Manafort's attorneys regularly briefed the President's lawyers on the topics discussed and the information ManaforthadprovidedininterviewswiththeSpecialCounsel'sOffice.*®° On November26, 2018, the Special Counsel's Office disclosed in a public court filing that Manafort had breached his plea agreement by lying about multiple subjects.**° The next day, Giuliani said that the President had been “upset for weeks” about what he considered to be “the un-American, horrible treatment of Manafort.”®8” In an interview on November 28, 2018, the President suggested that it was “very brave” that Manafort did not “flip”:

Ifyoutoldthetruth, yougotojail. Youknowthisflippingstuffisterrible. Youflipand youlieandyouget—theprosecutorswilltellyou99percentofthetimetheycangetpeople to flip. It's rare that they can't. But I had three people: Manafort, Corsi—l don't know
Corsi, but he refuses to say what they demanded.*** Manafort, actually very brave.*®

In response to a question abouta potential pardon for Manafort, the President said, “It was never discussed, butIwouldn'ttakeitoffthetable. WhywouldItakeitoffthetable?”*”°

\subsubsection{[████████: Harm to Ongoing Matter]}

\blackout{Harm to Ongoing Investigation}% 891
\blackout{Harm to Ongoing Investigation}% 892
\blackout{Harm to Ongoing Investigation}% 893

\blackout{Harm to Ongoing Investigation}% 894
\blackout{Harm to Ongoing Investigation}% 895
\blackout{Harm to Ongoing Investigation}% 896

\blackout{Harm to Ongoing Investigation}

\blackout{Harm to Ongoing Investigation}% 897
\blackout{Harm to Ongoing Investigation}% 898
\blackout{Harm to Ongoing Investigation}% 899
\blackout{Harm to Ongoing Investigation}% 900

\blackout{Harm to Ongoing Investigation}% 900
\blackout{Harm to Ongoing Investigation}% 900
\blackout{Harm to Ongoing Investigation}% 900

\blackout{Harm to Ongoing Investigation}% 900

\blackout{Harm to Ongoing Investigation}% 900
\blackout{Harm to Ongoing Investigation}% 900
\blackout{Harm to Ongoing Investigation}% 900
\blackout{Harm to Ongoing Investigation}% 900

Analysis

InanalyzingthePresident'sconducttowardsFlynn, Manafort, Roa.thefollowing evidence is relevant to the elements of obstruction of justice:

Obstructive act. The President's actions towards witnesses in the Special Counsel's investigation would qualify as obstructive if they had the natural tendency to prevent particular witnesses from testifying truthfully, or otherwise would have the probable effect of influencing, delaying, or preventing their testimony to law enforcement.

With regard to Flynn, the President sent private and public messages to Flynn encouraging him to stay strong and conveying that the President still cared about him before he began to cooperate with the government. When Flynn's attorneys withdrew him from a joint defense agreement with the President, signaling that Flynn was potentially cooperating with the government, the President's personal counsel initially reminded Flynn's counsel of the President's warm feelings towards Flynn and said “that still remains.” But when Flynn's counsel reiterated that Flynn could no longer share information under a joint defense agreement, the President's personal counsel stated that the decision would be interpreted as reflecting Flynn's hostility towards the President. That sequence of events could have had the potential to affect Flynn's decision to cooperate, as well as the extent of that cooperation. Because of privilege issues, however, we could not determine whether the President was personally involved in or knew about the specific message his counsel delivered to Flynn's counsel.

With respect to Manafort, there is evidence that the President's actions had the potential to influence Manafort's decision whether to cooperate with the government. The President and his personal counsel made repeated statements suggesting that a pardon wasa possibility for Manafort, while also making it clear that the President did not want Manafort to “flip” and cooperate with the government. On June 15, 2018, the day the judge presiding over Manafort's D.C. case was considering whether to revokehisbail, the President said that he “felt badly” for Manafort and stated, “I think a lot of it is very unfair.” And when asked about a pardon for Manafort, the President said, “I do want to see people treated fairly. That's what it's all about.” Later that day, after Manafort's bail was revoked, the President called it a “tough sentence” that was “Very unfair!” Two days later, the President's personal counsel stated that individuals involved in the Special Counsel's investigation could receive a pardon“if in fact the [P]resident and his advisors ... come to the conclusion that you have been treated unfairly”'—using language that paralleled how the President had already described the treatment of Manafort. Those statements, combined withthePresident'scommendationofManafortforbeinga“brave man”who“refused to“break',” suggested that a pardon was a more likely possibility if Manafort continued not to cooperate with the government. And while Manafort eventually pleaded guilty pursuant to a cooperation agreement, he was found to have violated the agreementbylying to investigators.

The President's public statements during the Manafort trial, including during jury deliberations, also had the potential to influence the trial jury. On the second dayoftrial, for example, the President called the prosecution a “terrible situation” and a “hoax”that “continues to stain our country” and referred to Manafort as a “Reagan/Doledarling” who was“serving solitary confinement” even though he was “convicted of nothing.” Those statements were widely picked up by the press. While jurors were instructed not to watch or read news stories about the case and are presumed to follow those instructions, the President's statements during the trial generated substantialmediacoveragethatcouldhavereachedjurorsiftheyhappenedtoseethestatements or learned about them from others. And the President's statements during jury deliberations that Manafort “happens to be a very good person” and that“it's very sad what they've done to Paul Manafort”hadthepotentialtoinfluencejurorswholearnedofthestatements, whichthePresident made just as jurors were considering whether to convict or acquit Manafort.

\blackout{Harm to Ongoing Investigation}

Nexus to an official proceeding. The President's actions towards Flynn, Manafort, appear to have been connected to pendingoranticipated official proceedings involving each individual. ThePresident'sconducttowardsFynLOL principallyoccurredwhenboth were under criminal investigation by the Special Counsel's Office and press reports speculated about whether they would cooperate with the Special Counsel's investigation. And the President's conduct towards Manafort was directly connected to the official proceedings involving him. The President made statements about Manafort and the charges against him during Manafort's criminal
trial. And the President's comments about the prospect of Manafort “flipping” occurred when it was clear the Special Counsel continued to oversee grand jury proceedings.

Intent. Evidence concerning the President's intent related to Flynn as a potential witness is inconclusive. As previously noted, becauseofprivilege issues we do not have evidence establishing whether the President knew about or wasinvolvedin his counsel's communications with Flynn's counsel stating that Flynn's decision to withdraw from the joint defense agreement and cooperate with the government would be viewed as reflecting “hostility” towards the President. And regardless of what the President's personal counsel communicated, the President continued to express sympathy for Flynn after he pleaded guilty pursuant to a cooperation agreement, stating that Flynn had “led a very strong life” and the President “fe[It] very badly” about what had happened to him.

Evidence concerning the President's conduct towards Manafort indicates that the President intended to encourage Manafort to not cooperate with the government. Before Manafort was convicted, thePresidentrepeatedlystatedthatManaforthadbeentreatedunfairly. Onedayafter Manafort was convicted on eight felony charges and potentially faced a lengthy prison term, the PresidentsaidthatManafortwas“abraveman”forrefusingto“break”and that“flipping”“almost ought to be outlawed.” At the same time, although the President had privately told aides he did not like Manafort, he publicly called Manafort “a good man”and said he had a “wonderful family.” And when the President was asked whether he was considering a pardon for Manafort, the President did not respond directly and instead said he had “great respect for what [Manafort]'s done, intermsofwhathe'sgonethrough.” ThePresidentaddedthat“someofthechargesthey threw against him, every consultant, every lobbyist in Washington probably does.” In light of the President's counsel's previous statements that the investigations “might get cleaned up with some presidential pardons” and that a pardon would be possible if the President “come[s] to the conclusion that you have been treated unfairly,” the evidence supports the inference that the President intended Manafort to believe that he could receive a pardon, which would make cooperation with the government as a meansofobtaining a lesser sentence unnecessary.

We also examined the evidenceofthe President's intent in making public statements about Manafortatthebeginningofhistrialandwhenthejurywasdeliberating. Someevidencesupports a conclusion that the President intended, at least in part, to influence the jury. The trial generated widespread publicity, and as the jury began to deliberate, commentators suggested that an acquittal would add to pressure to end the Special Counsel's investigation. By publicly stating on the second day of deliberations that Manafort “happens to be a very good person”and that“it's very sad what they've done to Paul Manafort” right after calling the Special Counsel's investigation a “rigged witch hunt,” the President's statements could, if they reached jurors, have the natural tendency to engender sympathy for Manafort among jurors, and a fact finder could infer that the President intended that result. But there are alternative explanations for the President's comments, including that he genuinely felt sorry for Manafort or that his goal was not to influence the jury but to influence public opinion. The President's comments also could have been intended to continue sending a message to Manafort that a pardon was possible. As described above, the President made his comments about Manafort being “a very good person” immediately after declining to answera question about whether he would pardon Manafort.

\blackout{Harm to Ongoing Investigation}

\subsection{The President's Conduct Involving Michael Cohen}

Overview

The President's conduct involving Michael Cohen spans the full period of our investigation. During the campaign, Cohen pursued the Trump Tower Moscow project on behalf of the Trump Organization. Cohen briefed candidate Trump on the project numerous times, including discussing whether Trump should travel to Russia to advance the deal. After the media began questioning Trump's connections to Russia, Cohen promoted a “party line” that publicly distanced Trump from Russia and asserted he had no business there. Cohen continued to adhere to that party line in 2017, when Congress asked him to provide documents and testimony in its Russia investigation. In an attempt to minimize the President's connections to Russia, Cohen submitted a letter to Congress falsely stating that he only briefed Trump on the Trump Tower Moscowprojectthree times, that he did not consider asking Trump to travel to Russia, that Cohen had not received a response to an outreach he made to the Russian government, and that the project ended in January 2016, before the first Republican caucus or primary. While working on the congressional statement, Cohen had extensive discussions with the President's personal counsel, who, according to Cohen, said that Cohen should not contradict the President and should keep the statement short and “tight.” After the FBI searched Cohen's home and office in April 2018, the President publicly asserted that Cohen would not “flip” and privately passed messages of support to him. Cohen also discussed pardons with the President's personal counsel and believed that if he stayed on message, he would get a pardon or the President would do “something else” to make
the investigation end. But after Cohen began cooperating with the government in July 2018, the President publicly criticized him, called him a “rat,” and suggested his family members had committed crimes.

Evidence

\subsubsection{Candidate Trump's Awareness of and Involvement in the Trump Tower Moscow Project}

The President's interactions with Cohen as a witness took place against the background of the President's involvement in the Trump Tower Moscow project.

As described in detail in Volume I, Section IV.A.1, supra, from September 2015 until at least June 2016, the Trump Organization pursued a Trump Tower Moscow project in Russia, with negotiations conducted by Cohen, then-executive vice president of the Trump Organization and special counsel to Donald J. Trump.%?
The Trump Organization had previously and unsuccessfully pursued a building project in Moscow.”!° According to Cohen, in approximately September 2015 he obtained internal approval from Trump to negotiate on behalf of the Trump Organization to have a Russian corporation build a tower in Moscowthatlicensed the Trump name and brand.°'! CohenthereafterhadnumerousbriefconversationswithTrumpabouttheproject.?”? CohenrecalledthatTrumpwantedtobeupdatedonanydevelopmentswithTrumpTowerMoscow and on several occasions brought the project up with Cohen to ask what was happening on it." Cohen also discussed the project on multiple occasions with Donald Trump Jr. and Ivanka Trump.?!4

In the fall of 2015, Trump signed a Letter of Intent for the project that specified highly lucrative terms for the Trump Organization.?'> In December 2015, Felix Sater, who was handling negotiations between Cohen and the Russian corporation, asked Cohen for a copy of his and Trump's passports to facilitate travel to Russia to meet with government officials and possible financing partners.?'© Cohen recalled discussing the trip with Trump and requesting a copy of Trump's passport from Trump's personal secretary, Rhona Graff?!”

By January 2016, Cohen had become frustrated that Sater had not set up a meeting with Russian government officials, so Cohen reached out directly by email to the office of Dmitry Peskov, whowasPutin'sdeputychiefofstaffandpresssecretary.°'* On January20, 2016, Cohen received an email response from Elena Poliakova, Peskov's personal assistant, and phone records confirm that they then spoke for approximately twenty minutes, during which Cohen described the Trump Tower Moscow project and requested assistance in moving the project forward.?!? Cohen recalled briefing candidate Trump about the call soon afterwards.°”° Cohen told Trump he spoke withawomanheidentifiedas“someonefromtheKremlin,”andCohenreportedthatshewasvery professionalandaskeddetailedquestionsabouttheproject."' CohenrecalledtellingTrumphe wished the Trump Organization had assistants who were as competent as the woman from the Kremlin?”

Cohenthoughthisphonecallrenewedinterestintheproject.“ ThedayafterCohen'scall with Poliakova, Sater texted Cohen, asking him to “[c]all me when you have a few minutes to chat ...It'saboutPutintheycalledtoday.”°* SatertoldCohenthattheRussiangovernmentlikedthe project and on January 25, 2016, sent an invitation for Cohen to visit Moscow “for a working visit.”°2> After the outreach from Sater, Cohenrecalledtelling Trump that he was waiting to hear back on moving the project forward.”

After January2016, CohencontinuedtohaveconversationswithSateraboutTrumpTower Moscow and continued to keep candidate Trump updated about those discussions and the status oftheproject.°?” CohenrecalledthatheandTrumpwantedTrumpTowerMoscowtosucceedand that Trump never discouraged him from working on the project because of the campaign.?* In March or April 2016, Trump asked Cohen if anything was happening in Russia.” Cohen also recalled briefing Donald Trump Jr. in the spring—a conversation that Cohen said was not “idle chit chat” because Trump Tower Moscow was potentially a \$1 billion deal.”°

Cohen recalled that around May 2016, he again raised with candidate Trumpthepossibility ofatriptoRussiatoadvancetheTrumpTowerMoscowproject.”*' Atthattime, Cohen had receivedseveraltextsfromSaterseekingtoarrangedatesforsuchatrip.”On May4, 2016, Sater wrote to Cohen, “I had a chat with Moscow. ASSUMINGthetrip does happen the question is beforeoraftertheconvention.....Obviouslythepremeetingtrip(you only)canhappenanytime you want but the 2 big guys[is] the question. I said I would confirm and revert.”'*? Cohen responded, “My trip before Cleveland. Trump once he becomes the nominee after the convention.”**4 On May 5, 2016, Sater followed up with a text that Cohen thought he probably read to Trump:

Peskov would like to invite you as his guest to the St. Petersburg Forum which is Russia's Davos it's June 16-19. He wants to meet there with you and possibly introduceyouto either Putin or Medvedev. ... This is perfect. The entire business classofRussiawillbethereaswell. Hesaidanythingyouwanttodiscussincluding dates and subjects are on the table to discuss.”

Cohen recalled discussing the invitation to the St. Petersburg Economic Forum with candidate Trump and saying that Putin or Russian Prime Minister Dmitry Medvedev might be there.CohenrememberedthatTrumpsaidthathewouldbewillingtotraveltoRussiaifCohen could “lock and load” on the deal.' In June 2016, Cohen decided not to attend the St. Petersburg Economic Forum because Sater had not obtained a formal invitation for Cohen from Peskov.?** Cohen said he had a quick conversation with Trumpatthat time but did not tell him that the project was over because he did not want Trump to complain that the deal was on-again-off-again if it were revived.”

During the summer of 2016, Cohen recalled that candidate Trump publicly claimed that he had nothing to do with Russia and then shortly afterwards privately checked with Cohen about the status of the Trump Tower Moscow project, which Cohen found “interesting.”°4° At some point that summer, Cohen recalled having a brief conversation with Trump in which Cohen said the Trump Tower Moscow project was going nowhere because the Russian development company hadnotsecuredapieceofpropertyfortheproject.! Trumpsaidthatwas“too bad,”and Cohen did not recall talking with Trumpaboutthe project after that.” Cohen said that at no time during the campaign did Trump tell him not to pursue the project or that the project should be abandoned.”

\subsubsection{Cohen Determines to Adhere to a “Party Line” Distancing Candidate Trump From Russia}

As previously discussed, see Volume II, Section II.A, supra, when questions about possible Russian support for candidate Trump emerged during the 2016 presidential campaign, Trump denied having any personal, financial, or business connection to Russia, which Cohen described as the “party line”or “message”to follow for Trump and his senior advisors.*"

After the election, the Trump Organization sought to formally close out certain deals in advanceoftheinauguration.”4° CohenrecalledthatTrumpTowerMoscowwasonthelistofdeals to be closed out.“ In approximately January 2017, Cohen began receiving inquiries from the media about Trump Tower Moscow, and he recalled speaking to the President-Elect when those inquiries came in.*” Cohen was concernedthattruthful answers about the Trump Tower Moscow project might not be consistent with the “message” that the President-Elect had no relationship with Russia.“*

In an effort to “stay on message,” Cohen told a New York Times reporter that the Trump Tower Moscow deal wasnotfeasible and had ended in January 2016.Cohen recalled that this waspartofa“script”ortalkingpointshehaddevelopedwithPresident-ElectTrumpandothersto
dismiss the idea of a substantial connection between Trump and Russia.” Cohen said that he discussed the talking points with Trump but that he did not explicitly tell Trump he thought they wereuntruebecauseTrumpalreadyknewtheywereuntrue.”°! Cohenthoughtitwasimportantto say the deal was done in January 2016, rather than acknowledge that talks continued in May and June 2016, because it limited the period when candidate Trump could be alleged to have a relationship with Russia to an early point in the campaign, before Trump had become the party's presumptive nominee.'

\subsubsection{Cohen Submits False Statements to Congress Minimizing the Trump Tower Moscow Project in Accordance with the Party Line}

In early May 2017, Cohen received requests from Congress to provide testimony and documents in connection with congressional investigations of Russian interference in the 2016 election? At that time, Cohen understood Congress's interest in him to be focused on the allegations in the Steele reporting concerning a meeting Cohen allegedly had with Russian officials in Prague during the campaign.?* Cohen had never traveled to Prague and was not concerned about those allegations, which he believed were provably false.9°° On May 18, 2017, Cohen met with the President to discuss the request from Congress, and the President instructed Cohen that he should cooperate because there was nothing there.?>°

Cohen eventually entered into ajoint defense agreement (JDA)with the President and other individuals who were part of the Russia investigation.” In the months leading up to his congressional testimony, Cohen frequently spoke with the President's personal counsel.** Cohen said that in those conversations the President's personal counsel would sometimes say that he had just been with the President.Cohen recalled that the President's personal counsel told him the JDA was working well together and assured him that there was nothing there and if they stayed on messagetheinvestigationswouldcometoanendsoon.Atthattime, Cohen'slegalbillswere being paid by the Trump Organization,*®! and Cohen was told not to worry because the investigations would be over by summer or fall of 2017.Cohen said that the President's personal counsel also conveyed that, as part of the JDA, Cohen was protected, which he would not be if he “went rogue.”* Cohen recalled that the President's personal counsel reminded him that “the President loves you”and told him that if he stayed on message, the President had his back.°TM

In August2017, CohenbegandraftingastatementaboutTrumpTowerMoscowtosubmit to Congress along with his document production.” The final version of the statement contained several false statements about the project.°First, although the Trump Organization continued to pursue the project until at least June 2016, the statement said, “The proposal was under consideration at the Trump Organization from September 2015 until the end of January 2016. By the end of January 2016, I determined that the proposal was not feasible for a variety of business reasons and should not be pursued further. Based on my business determinations, the Trump Organization abandoned the proposal.”°6” Second, although Cohen and candidate Trump had discussed possible travel to Russia by Trump to pursue the venture, the statement said, “Despite overturesbyMr.Sater, IneverconsideredaskingMr.TrumptotraveltoRussiainconnectionwith this proposal. I told Mr. Sater that Mr. Trump would not travel to Russia unless there was a definitive agreement in place.”*Third, although Cohen had regularly briefed Trump onthestatus of the project and had numerous conversationsaboutit, the statement said, “Mr. Trump was never in contact with anyone about this proposal other than me on three occasions, including signing a non-binding letter of intent in 2015.”Fourth, although Cohen's outreach to Peskov in January 2016 had resulted in a lengthy phone call with a representative from the Kremlin, the statement said that Cohen did “not recall any response to my email [to Peskov], nor any other contacts by me with Mr. Peskov or other Russian government officials about the proposal.”°”°

Cohen's statement was circulated in advance to, and edited by, members of the JDA.7”! Before the statement was finalized, early drafts contained a sentence stating, “The building project led me to make limited contacts with Russian government officials.”” In the final version of the statement, that line was deleted.””? Cohen though the was told that it was a decision of the JDA to take out that sentence, and he did not push back on the deletion.°'* Cohen recalled that he told the President'spersonalcounselthathewouldnotcontestadecisionoftheJDA.””

Cohen also recalled that in drafting his statement for Congress, he spoke with the President's personal counsel about a different issue that connected candidate Trump to Russia: Cohen's efforts to set up a meeting between TrumpandPutin in New York during the 2015 United Nations General Assembly.?”° In September 2015, Cohen had suggested the meeting to Trump, who told Cohen to reach out to Putin's office about it.°”” Cohen spoke and emailed with a Russian official about a possible meeting, and recalled that Trump asked him multiple times for updates on the proposed meeting with Putin.°”® When Cohen called the Russian official a second time, she told him it would not follow properprotocolfor Putin to meet with Trump, and Cohen relayed that message to Trump.?”? Cohen anticipated he might be asked questions about the proposed Trump- Putin meeting when he testified before Congress because he had talked about the potential meeting on Sean Hannity's radio show.°®° Cohen recalled explaining to the President's personal counsel the “whole story” of the attempt to set up a meeting between Trump and Putin and Trump's role in it.°8! CohenrecalledthatheandthePresident'spersonalcounseltalkedaboutkeepingTrump out of the narrative, and the President's personal counsel told Cohen the story was not relevant and should not be included in his statement to Congress.?*

Cohen said that his “agenda” in submitting the statement to Congress with false representations about the Trump Tower Moscow project was to minimize links between the project and the President, give the false impression that the project had ended before the first presidential primaries, and shut down further inquiry into Trump Tower Moscow, with the aim of limiting the ongoingRussiainvestigations.”*? CohensaidhewantedtoprotectthePresidentandbeloyalto him by not contradicting anything the President had said.?** Cohen recalled he was concerned that if he told the truth about getting a response from the Kremlin or speaking to candidate Trump about travel to Russia to pursue the project, he would contradict the message that no connection existed between Trump and Russia, and he rationalized his decision to provide false testimony because the deal never happened.?® He was not concerned that the story would be contradicted by individualswhoknewitwasfalsebecausehewasstickingtothepartylineadheredtobythewhole group.°\%° Cohen wanted the support of the President and the White House, and he believed that following the party line would help put an end to the Special Counsel and congressional investigations.?*”

Between August 18, 2017, when the statement was in an initial draft stage, and August 28, 2017, when the statement was submitted to Congress, phone records reflect that Cohen spoke with the President's personal counsel almost daily.?8* On August 27, 2017, the day before Cohen submitted the statement to Congress, Cohen and the President's personal counsel had numerous contacts by phone, including calls lasting three, four, six, eleven, and eighteen minutes.8? Cohen recalled telling the President's personal counsel, who did not have first-hand knowledge of the project, that there was more detail on Trump Tower Moscow that was not in the statement, including that there were more communications with Russia and more communications with candidate Trump than the statement reflected.°” Cohen stated that the President's personal counsel responded that it was not necessary to elaborate or include those details because the project did not progress and that Cohen should keep his statement short and “tight” and the matter would soon come to an end.””! Cohen recalled that the President's personal counsel said “his client” appreciated Cohen, that Cohen should stay on message and not contradict the President, that there wasnoneedtomuddythewater, andthatitwastimetomoveon.”? Cohensaidheagreedbecause itwaswhathewasexpectedtodo.AfterCohenlaterpleadedguiltytomakingfalsestatements to Congress about the Trump Tower Moscow project, this Office sought to speak with the President's personal counsel about these conversations with Cohen, but counsel declined, citing potential privilege concerns.°*

At the same time that Cohen finalized his written submission to Congress, he served as a source for a Washington Post story published on August 27, 2017, that reported in depth for the first time that the Trump Organization was “pursuing a plan to develop a massive Trump Tower in Moscow”at the same time as candidate Trump was“running for president in late 2015 and early 2016.”° The article reported that “the project was abandoned at the end of January 2016, just beforethepresidentialprimariesbegan, severalpeoplefamiliarwiththeproposalsaid.””°° Cohen recalled that in speaking to the Post, he held to the false story that negotiations for the deal ceased in January 2016.”

On August 28, 2017, Cohen submitted his statement about the Trump Tower Moscow projecttoCongress.”* CohendidnotrecalltalkingtothePresidentaboutthespecificsofwhat the statement said or what Cohen would later testify to about Trump Tower Moscow.” He recalled speaking to the President more generally about how he planned to stay on message in his testimony.' On September 19, 2017, in anticipation of his impending testimony, Cohen orchestrated the public release of his opening remarks to Congress, which criticized the allegations in the Steele material and claimed that the Trump Tower Moscow project “was terminated in January of 2016; which occurred before the Iowa caucus and months before the very first primary.”!°°! Cohensaidthereleaseofhisopeningremarkswasintendedtoshapethenarrative and let other people who might be witnesses know what Cohen was saying so they could follow the same message.'? Cohen said his decision was meant to mirror Jared Kushner's decision to release a statement in advance of Kushner's congressional testimony, which the President's personal counsel had told Cohen the President liked.!°°? Cohen recalled that on September 20, 2017, after Cohen's opening remarks had been printed by the media, the President's personal counsel told him that the President was pleased with the Trump Tower Moscowstatementthat had gone out,!0%

On October 24 and 25, 2017, Cohen testified before Congress and repeated the false statements he had included in his written statement about Trump Tower Moscow.!°° Phone records show that Cohen spoke with the President's personal counsel immediately after his testimony on both days.!°°%

\subsubsection{The President Sends Messages of Support to Cohen}

In January 2018, the media reported that Cohen had arranged a \$130, 000 payment during the campaign to prevent a woman from publicly discussing an alleged sexual encounter she had with the President before he ran for office.'°” This Office did not investigate Cohen's campaign- period payments to women.'°? However, those events, as described here, are potentially relevant to the President's and his personal counsel's interactions with Cohen as a witness who later began to cooperate with the government.

On February 13, 2018, Cohen released a statement to news organizations that stated, “In a
private transaction in 2016, I used my own personalfundsto facilitate a payment of \$130, 000 to
[the woman]. Neither the Trump Organization nor the Trump campaign was a party to the
May-Contain-
transaction with [the woman], and neither reimbursed me for the payment, either directly or
indirectly.”IncongressionaltestimonyonFebruary27, 2019, Cohentestifiedthathehad
discussed what to say about the payment with the President and that the President had directed
Cohen to say that the President “was not knowledgeable . . . of [Cohen's] actions” in making the
payment.'°!© Qn February 19, 2018, the day after the New York Times wrote a detailed story
attributing the payment to Cohen and describing Cohen as the President's “fixer,” Cohen received
a text message from the President's personal counsel that stated, “Client says thanks for what you do.7!0l!

On April 9, 2018, FBI agents working with the U.S. Attorney's Office for the Southern District of New York executed search warrants on Cohen's home, hotel room, and office.'°'? That day, thePresidentspoketoreportersandsaidthathehad“justheardthattheybrokeintotheoffice of one of my personal attorneys—a good man.”'°'3 The President called the searches “a real disgrace” and said, “It's an attack on our country, in a true sense. It's an attack on what we all stand for.”!°'* Cohen said that after the searches he was concerned that he was “an open book,” that he did not want issues arising from the payments to women to “come out,” and that his false statements to Congress were ‘‘a big concern.”!°!°

A few days after the searches, the President called Cohen.'"! According to Cohen, the President said he wanted to “check in” and asked if Cohen was okay, and the President encouraged Cohen to “hang in there”and“stay strong.”!°'? Cohen also recalled that following the searches he heard from individuals who we rein touch with the President and relayed to Cohen the President's
a.
supportforhim.'°!* Cohenrecalledthat
to say that he was with “the Boss” in Mar-a-Lago and the President had said “he loves you” and not to worry.'°!? Cohen recalled that for the Trump Organization, told him,“the boss loves you.” , a friend of the President's, told him, “everyone knows the boss has your back.”

On or about April 17, 2018, Cohen began speaking with an attorney, Robert Costello, who had a close relationship with Rudolph Giuliani, one of the President's personal lawyers.idee Costello told Cohen that he had a “back channel of communication” to Giuliani, and that Giuliani hadsaidthe“channel”was“crucial”and“mustbemaintained.”!°On April20, 2018, the New York Times published an article about the President's relationship with and treatment of Cohen.'TM The President responded with a series of tweets predicting that Cohen would not“flip”:

The New York Times and third rate reporter . . . are going out of their way to destroy Michael Cohenandhis relationship with me in the hope that he will ‘flip.' They use non- existent ‘sources' and a drunk/drugged up loser who hates Michael, a fine person with a wonderful family. Michael is a businessman for his own account/lawyer whoI have always liked&respected. MostpeoplewillflipiftheGovernmentletsthemoutoftrouble, even if it means lying or making up stories. Sorry, I don't see Michael doing that despite the horrible Witch Huntandthe dishonest media!!©%

In an email that day to Cohen, Costello wrote that he had spoken with Giuliani.'© Costello told Cohen the conversation was “Very Very Positive[.] You are ‘loved'. . . they are in our corner. .. . Sleep well tonight[], you have friends in high places.”!°77

Cohen said that following these messages he believed he had the support of the White House if he continued to toe the party line, and he determined to stay on message and be part of the team.!°28 At the time, Cohen's understood that his legal fees were still being paid by the Trump Organization, whichhesaidwasimportanttohim.'°? Cohenbelievedheneededthepowerofthe Presidenttotakecareofhim, soheneededtodefendthePresidentandstayonmessage.'©°

Cohen also recalled speaking with the President's personal counsel about pardons after the searches of his home and office had occurred, at a time when the media had reported that pardon discussionswereoccurringattheWhiteHouse.'°3! CohentoldthePresident'spersonalcounsel he had been a loyal lawyer and servant, and he said that after the searches he was in an uncomfortable position and wanted to know what was in it for him.'%?
According to Cohen, the President's personal counsel responded that Cohen should stay on message, that the investigation was a witch hunt, and that everything would be fine.'? Cohen understood based on this conversation and previous conversations about pardons with the President's personal counsel that as long as he stayed on message, he would be taken care of by the President, either through a pardonorthrough the investigation being shut down.'°*

On April 24, 2018, the President responded to a reporter's inquiry whether he would consider a pardon for Cohen with, “Stupid question.”'> On June 8, 2018, the President said he “hadn't even thought about” pardons for Manafort or Cohen, and continued,“It's far too early to be thinking about that. They haven't been convicted of anything. There's nothing to pardon.”!°*° And on June 15, 2018, the President expressed sympathy for Cohen, Manafort, and Flynn in a press interview and said, “I feel badly about a lot of them, because I think lot of it is very unfair.”!37

\subsubsection{The President's Conduct After Cohen Began Cooperating with the Government}

On July 2, 2018, ABC News reported based on an “exclusive” interview with Cohen that Cohen “strongly signaled his willingness to cooperate with special counsel Robert Mueller and federalprosecutorsintheSouthernDistrictofNewYork—evenifthatputsPresidentTrumpin
jeopardy.”'°35That week, themediareportedthatCohenhadaddedanattorneytohislegalteam who previously had workedasa legal advisor to President Bill Clinton.'°°

Beginning on July 20, 2018, the media reported on the existence of a recording Cohen had made of a conversation he had with candidate Trump about a payment made to a second woman who said she had had an affair with Trump.'°4? On July 21, 2018, the President responded: “Inconceivable that the government would break into a lawyer's office (early in the morning}— almost unheard of. Even more inconceivable that a lawyer would tape a client—totally unheard of & perhaps illegal. The good news is that your favorite President did nothing wrong!”""' On July 27, 2018, after the media reported that Cohen was willing to inform investigators that Donald Trump Jr.told his father about the June 9, 2016 meeting to get “dirt” on Hillary Clinton,'the President tweeted: “[S]o the Fake News doesn't waste my time with dumb questions, NO, I did NOTknowofthemeetingwithmyson, Don jr. Soundstomelikesomeoneistryingtomakeup stories in order to get himself out of an unrelated jam (Taxi cabs maybe?). He even retained Bill and Crooked Hillary's lawyer. Gee, I wonder if they helped him make the choice!”

On August 21, 2018, Cohen pleaded guilty in the Southern District of New York to eight felony charges, including two counts of campaign-finance violations based on the payments he had made during the final weeks of the campaign to women who said they had affairs with the President.!°* During the plea hearing, Cohen stated that he had worked“at the direction of” the candidate in making those payments.'“° The next day, the President contrasted Cohen's cooperation with Manafort's refusal to cooperate, tweeting, “I feel very badly for Paul Manafort and his wonderful family. ‘Justice' took a 12 year old tax case, among other things, applied tremendous pressure on him and, unlike Michael Cohen, he refused to ‘break'—make up stories inordertogeta‘deal.' Suchrespectforabraveman!”!%°

On September 17, 2018, this Office submitted written questions to the President that included questions about the Trump Tower Moscow project and attached Cohen's written statementtoCongressandtheLetterofIntentsignedbythePresident.'°*' Amongotherissues, the questions asked the President to describe the timing and substance of discussions he had with Cohen about the project, whether they discussed a potential trip to Russia, and whether the President“at any time direct[ed] or suggest[ed] that discussions about the Trump Moscow project should cease,” or whether the President was “informed at any time that the project had been abandoned.”!%48

On November20, 2018, the President submitted written responses that did not answer those questions about Trump Tower Moscow directly and did not provide any information about the timingofthecandidate'sdiscussionswithCohenabouttheprojectorwhetherheparticipatedin any discussions about the project being abandoned or no longer pursued.'*? Instead, the President's answers stated in relevant part:

I had few conversations with Mr. Cohen on this subject. AsI recall, they were brief, and they were not memorable. I was not enthused about the proposal, and I do not recall any discussionoftraveltoRussiainconnectionwithit. Idonotrememberdiscussingitwith anyone else at the Trump Organization, although it is possible. I do not recall being aware at the time of any communications between Mr. Cohen and Felix Sater and any Russian government official regarding the Letter of Intent.!°

On November 29, 2018, Cohen pleaded guilty to making false statements to Congress basedonhisstatementsabouttheTrumpTowerMoscowproject.'®!' Inapleaagreementwiththis Office, Cohen agreed to “provide truthful information regarding any and all matters as to which this Office deems relevant.”!°** Later on November29, after Cohen's guilty plea had become public, the President spoke to reporters about the Trump Tower Moscow project, saying:

I decided not to do the project... . I decided ultimately not to do it. There would have beennothingwrongifIdiddoit. IfIdiddoit, therewouldhavebeennothingwrong. That
May-Centain-
ItwasanoptionthatIdecidednottodo.... Idecidednottodoit. . I was running my
was my business...
The primary reason... I was focused on running for President...
business while I was campaigning. There was a good chance that I wouldn't have won, in which case I would've gone back into the business. And why should I lose lots of opportunities?!°

ThePresidentalso said that Cohen was “a weak person. And by being weak, unlike other people that you watch—he is a weak person. And what he's trying to do is get a reduced sentence. So he'slyingaboutaprojectthateverybodyknewabout.”!°* ThePresidentalsobroughtupCohen's written submission to Congress regarding the Trump Tower Moscow project: “So here's the story: Go back and look at the paper that Michael Cohen wrote before he testified in the House and/or Senate. It talked about his position.”!°°° The President added, “Even if [Cohen] was right, it doesn't matter because I was allowed to do whateverI wanted during the campaign.”!°°6

In light of the President's public statements following Cohen's guilty plea that he “decided not to do the project,” this Office again sought information from the President about whether he participated in any discussions about the project being abandoned or no longer pursued, including when he“decidednottodotheproject,”whohespoketoaboutthatdecision, andwhatmotivated
the decision.'°' The Office also again asked for the timing of the President's discussions with Cohen about Trump Tower Moscow and asked him to specify “what period of the campaign” he was involved in discussions concerning the project.'8 In response, the President's personal counsel declined to provide additional information from the Presidentandstated that “the President has fully answered the questions at issue.”!°°?

In the weeks following Cohen's plea and agreement to provide assistance to this Office, the President repeatedly implied that Cohen's family members were guilty of crimes. On December 3, 2018, after Cohen had filed his sentencing memorandum, the President tweeted, ““Michael Cohen asks judge for no Prison Time.' You mean he can do all of the TERRIBLE, unrelated to Trump, things having to do with fraud, big loans, Taxis, etc., and not serve a long prison term? He makesupstories to get a GREAT & ALREADYreduced deal for himself, and get his wife and father-in-law (who has the money?) off Scott Free. He lied for this outcome and should, in my opinion, serveafullandcompletesentence.”!° \blackout{Harm to Ongoing Investigation}

On December12, 2018, Cohenwassentencedtothreeyearsofimprisonment.'° The next day, the President senta series of tweets that said:

IneverdirectedMichaelCohentobreakthelaw....Thosechargeswerejustagreedtoby him in order to embarrass the president and get a much reduced prison sentence, which he did—including the fact that his family was temporarily let off the hook. As a lawyer, Michaelhasgreatliability to me!!°

On December 16, 2018, the President tweeted, “Remember, Michael Cohen only becamea ‘Rat' after the FBI did something which was absolutely unthinkable \& unheard of until the Witch Hunt was illegally started. They BROKE INTO AN ATTORNEY'S OFFICE! Why didn't they break into the DNC to get the Server, or Crooked's office?”!°TM

In January 2019, after the media reported that Cohen would provide public testimony in a congressional hearing, the President made additional public comments suggesting that Cohen's family members had committed crimes. In an interview on Fox on January 12, 2019, the President was asked whether he was worried about Cohen's testimony and responded:

[{]n order to get his sentence reduced, [Cohen] says “I have an idea, I'll ah, tell—I'll give yousomeinformationonthepresident.” Well, thereisnoinformation. Butheshouldgive information maybeonhisfather-in-law because that's the one that people want to look at because where does that money—that's the money in the family. And I guess he didn't want to talk about his father-in-law, he's trying to get his sentence reduced. So it's ah, pretty sad. You know, it's weak and it's very sad to watch a thing like that.'°°

On January 18, 2019, the President tweeted, “Kevin Corke, @FoxNews ‘Don't forget, Michael Cohen has already been convicted of perjury and fraud, and as recently as this week, the Wall Street Journal has suggested that he mayhavestolen tens ofthousandsofdollars. . . .”- Lying toreducehisjailtime! Watch father-in-law!”|°°¢

On January 23, 2019, Cohen postponed his congressional testimony, citing threats against his family.'°\%7 Thenextday, thePresidenttweeted,“SointerestingthatbadlawyerMichaelCohen, who sadly will not be testifying before Congress, is using the lawyer of Crooked Hillary Clinton to represent him—Gee, how did that happen?”!°%

Also in January 2019, Giuliani gave press interviews that appeared to confirm Cohen's account that the Trump Organization pursued the Trump Tower Moscow project well past January 2016. Giuliani stated that “it's our understanding that [discussions about the Trump Moscow project] went on throughout 2016. Weren't a lot of them, but there were conversations. Can't be sure of the exact date. But the president can remember having conversations with him about it.
The president also remembers—yeah, probably up—could be up to as far as October, November.”!°® InaninterviewwiththeNewYorkTimes, GiulianiquotedthePresidentassaying that the discussions regarding the Trump Moscow project were “going on from the day I announced to the day I won.”!°7? On January 21, 2019, Giuliani issued a statement that said: “My recent statements about discussions during the 2016 campaign between Michael Cohen and candidate Donald Trump about a potential Trump Moscow ‘project' were hypothetical and not based on conversations I had with the president.”!07!

Analysis

In analyzing the President's conduct related to Cohen, the following evidence is relevant to the elements of obstruction of justice.

Obstructive act. WegatheredevidenceofthePresident'sconductrelatedtoCohen on two issues:(i) whether the President or others aided or participated in Cohen's false statements to Congress, and (ii) whether the President took actions that would have the natural tendency to prevent Cohen from providing truthful information to the government.

First, with regard to Cohen's false statements to Congress, while there is evidence, described below, that the President knew Cohen provided false testimony to Congress about the Trump Tower Moscow project, the evidence available to us does not establish that the President directed or aided Cohen's false testimony.

Cohen said that his statements to Congress followed a “party line” that developed within the campaign to align with the President's public statements distancing the President from Russia. Cohenalsorecalled that, in speaking with the President in advanceoftestifying, he made it clear that he would stay on message—which Cohen believed they both understood would require false testimony. But Cohen said that he and the President did not explicitly discuss whether Cohen's testimony about the Trump Tower Moscow project would be or was false, and the President did not direct him to provide false testimony. Cohen also said he did not tell the President about the specifics of his planned testimony. During the time when his statement to Congress was being drafted and circulated to members of the JDA, Cohen did not speak directly to the President about the statement, but rather communicated with the President's personal counsel—as corroborated by phone records showing extensive communications between Cohen and the President's personal counsel before Cohen submitted his statement and whenhetestified before Congress.

Cohen recalled that in his discussions with the President's personal counsel on August 27, 2017—the day before Cohen's statement was submitted to Congress—Cohen said that there were more communications with Russia and more communications with candidate Trump than the statement reflected. Cohen recalled expressing some concernatthat time. According to Cohen, the President's personal counsel—who did not have first-hand knowledge of the project— responded by saying that there was no need to muddy the water, that it was unnecessary to include those details because the project did not take place, and that Cohen should keep his statement short and tight, not elaborate, stay on message, and not contradict the President. Cohen's recollection of the content of those conversations is consistent with direction about the substance of Cohen's draft statement that appeared to come from members of the JDA. For example, Cohen omitted any reference to his outreach to Russian government officials to set up a meeting between Trump and Putin during the United Nations General Assembly, and Cohen believed it was a decision of theJDAtodeletethesentence,“ThebuildingprojectledmetomakelimitedcontactswithRussian government officials.”

The President's personal counsel declined to provide us with his account of his conversations with Cohen, and there is no evidence available to us that indicates that the President wasawareoftheinformationCohenprovidedtothePresident'spersonalcounsel. The President's conversations with his personal counsel were presumptively protected by attorney-client privilege, and we did not seek to obtain the contents of any such communications. The absence of evidence about the President and his counsel's conversations about the drafting of Cohen's statement precludes us from assessing what, if any, role the President played.

Second, we considered whether the President took actions that would have the natural tendency to prevent Cohen from providing truthful information to criminal investigators or to Congress.

Before Cohen began to cooperate with the government, the President publicly and privately urged Cohen to stay on message and not “flip.” Cohen recalled the President's personal counsel telling him that he would be protected so long as he did not go “rogue.” In the days and weeks thatfollowedtheApril2018searchesofCohen'shomeandoffice, thePresidenttoldreportersthat Cohenwasa “good man”and said he was“a fine person with a wonderful family ... who I have always liked & respected.” Privately, the President told Cohen to “hang in there” and “stay strong.” People who were close to both Cohen and the President passed messages to Cohen that “the President loves you,”“the boss loves you,” and “everyone knows the boss has your back.” Through the President's personal counsel, the President also had previously told Cohen “thanks for what you do” after Cohen provided information to the media about payments to women that, according to Cohen, both Cohen and the President knew was false. At that time, the Trump Organization continued to pay Cohen's legal fees, which was important to Cohen. Cohen also recalleddiscussingthepossibilityofapardonwiththePresident'spersonalcounsel, whotoldhim tostayonmessageandeverythingwouldbefine. ThePresidentindicatedinhispublicstatements that a pardon had not been ruled out, and also stated publicly that “[m]Jost people will flip if the Government lets them out of trouble”but that he “d[idn't] see Michael doing that.”

After it was reported that Cohen intended to cooperate with the government, however, the President accused Cohen of “mak[ing] up stories in order to get himself out of an unrelated jam (Taxi cabs maybe?),” called Cohen a “rat,” and on multiple occasions publicly suggested that Cohen's family members had committed crimes. The evidence concerning this sequence of events could support an inference that the President used inducements in the form of positive messages in an effort to get Cohen not to cooperate, and then turned to attacks andintimidation to deter the provision of information or undermine Cohen's credibility once Cohen began cooperating.

Nexus to an official proceeding. The President's relevant conduct towards Cohen occurred when the President knew the Special Counsel's Office, Congress, and the U.S. Attorney's Office for the Southern District of New York were investigating Cohen's conduct. The President acknowledged through his public statements and tweets that Cohen potentially could cooperate with the government investigations.

Intent. In analyzing the President's intent in his actions towards Cohen as a potential witness, there is evidence that could support the inference that the President intended to discourage Cohen from cooperating with the government because Cohen's information would shed adverse light on the President's campaign-period conduct and statements.

Cohen's false congressional testimony about the Trump Tower Moscow project was designed to minimize connections between the President and Russia and to help limit the congressional and DOJ Russia investigations—a goal that was in the President's interest, as reflected by the President's own statements. During and after the campaign, the President made repeated statements that he had “no business” in Russia and said that there were “no deals that could happen in Russia, because we've stayed away.” As Cohen knew, and asherecalled communicating to the President during the campaign, Cohen's pursuit of the Trump Tower Moscow project cast doubt on the accuracy or completeness of these statements.

In connection with his guilty plea, Cohen admitted that he had multiple conversations with candidate Trump to give him status updates about the Trump Tower Moscow project, that the conversations continued through at least June 2016, and that he discussed with Trump possible travel to Russia to pursue the project. The conversations were not off-hand, according to Cohen, because the project had the potential to be so lucrative. In addition, text messages to and from Cohen and other records further establish that Cohen's efforts to advance the project did not end in January 2016 and that in May and June 2016, Cohen was considering the timing for possible trips to Russia by him and Trump in connection with the project.

The evidence could support an inference that the President was aware of these facts at the timeofCohen'sfalsestatementstoCongress. CohendiscussedtheprojectwiththePresidentin early 2017 following media inquiries. Cohen recalled that on September 20, 2017, the day after hereleasedtothepublichisopeningremarkstoCongress—whichsaidtheproject“was terminated in January of 2016”—the President's personal counsel told him the President was pleased with what Cohen had said about Trump Tower Moscow. And after Cohen's guilty plea, the President told reporters that he had ultimately decided not to do the project, which supports the inference that he remained aware of his own involvement in the project and the period during the Campaign in which the project was' being pursued.

The President's public remarks following Cohen's guilty plea also suggest that the President may have been concerned about what Cohen told investigators about the Trump Tower Moscow project. At the time the President submitted written answers to questions from this Office about the project and other subjects, the media had reported that Cohen was cooperating with the government but Cohenhadnotyet pleaded guilty to making false statements to Congress. Accordingly, it was not publicly known what information about the project Cohen had provided to the government. In his written answers, the President did not provide details about the timing and substanceofhis discussions with Cohen about the project and gave no indication that he had decided to no longer pursue the project. Yet after Cohen pleaded guilty, the President publicly statedthathehadpersonallymadethedecisiontoabandontheproject. ThePresidentthendeclined to clarify the seeming discrepancy to our Office or answer additional questions. The content and timing of the President's provision of information about his knowledge and actions regarding the Trump Tower Moscow project is evidence that the President may have been concernedaboutthe
information that Cohen could provide as a witness.

The President's concern about Cohen cooperating may have been directed at the Southern District of New York investigation into other aspects of the President's dealings with Cohen rather than an investigation of Trump Tower Moscow. There also is some evidence that the President's concern about Cohen cooperating was based on the President's stated belief that Cohen would provide false testimony against the President in an attempt to obtain a lesser sentence for his unrelated criminal conduct. The President tweeted that Manafort, unlike Cohen, refused to “break” and “makeupstories in order to get a ‘deal.'” And after Cohen pleaded guilty to making false statements to Congress, the President said, “what [Cohen]'s trying to do is get a reduced sentence. So he's lying about a project that everybody knew about.” But the President also appeared to defend the underlying conduct, saying, “Even if [Cohen] was right, it doesn't matter because I was allowed to do whateverI wanted during the campaign.” As described above, there is evidence that the President knew that Cohen had made false statements about the Trump Tower Moscow project and that Cohen did so to protect the President and minimize the President's connections to Russia during the campaign.

Finally, the President's statements insinuating that members of Cohen's family committed crimes after Cohen began cooperating with the government could be viewed as an effort to retaliate against Cohen and chill further testimony adverse to the President by Cohen or others. It is possible that the President believes, as reflected in his tweets, that Cohen “ma[d]e[] up stories” in order to get a deal for himself and “get his wife and father-in-law .. . off Scott Free.” ItalsoispossiblethatthePresident'smentionofCohen'swifeandfather-in-lawwerenotintended toaffectCohenasawitnessbutratherwerepartofapublic-relationsstrategyaimedatdiscrediting Cohen and deflecting attention away from the President on Cohen-related matters. But the President's suggestion that Cohen's family members committed crimes happened more than once, includingjustbeforeCohenwassentenced(atthesametimeasthePresidentstatedthatCohen “should, in my opinion, serve a full and complete sentence”) and again just before Cohen was scheduled to testify before Congress. The timing of the statements supports an inference that they wereintendedatleast in part to discourage Cohen from further cooperation.

\subsection{Overarching Factual Issues}

Although this report does not contain a traditional prosecution decision or declination decision, the evidence supports several general conclusions relevant to analysis of the facts concerning the President's course of conduct.

Three features of this case render it atypical compared to the heartland obstruction-of- justice prosecutions brought by the DepartmentofJustice.

First, theconductinvolvedactionsbythePresident. Someoftheconductdidnotimplicate the President's constitutional authority and raises garden-variety obstruction-of-justice issues. Other events we investigated, however, drew upon the President's Article II authority, which raisedconstitutionalissuesthatweaddressinVolumeII, Section III-B, infra. Afactualanalysis of that conduct would have to take into account both that the President's acts were facially lawful and that his position as head of the Executive Branch provides him with unique and powerful meansofinfluencingofficial proceedings, subordinate officers, and potential witnesses.

Second, many obstruction cases involve the attempted or actual cover-up of an underlying crime. Personal criminal conduct can furnish strong evidence that the individual had an improper obstructive purpose, see, e.g., United States v. Willoughby, 860 F.2d 15, 24 (2d Cir. 1988), or that he contemplated an effect on an official proceeding, see, e.g., United States v. Binday, 804 F.3d 558, 591 (2d Cir. 2015). But proof of such a crime is not an element of an obstruction offense. See United States v. Greer, 872 F.3d 790, 798 (6th Cir. 2017)(stating, in applying the obstruction sentencing guideline, that “obstruction of a criminal investigation is punishable even if the prosecution is ultimately unsuccessful or even if the investigation ultimately reveals no underlying crime”). Obstruction of justice can be motivated by a desire to protect non-criminal personal interests, to protect against investigations where underlying criminal liability falls into a gray area, or to avoid personal embarrassment. The injury to the integrity of the justice system is the same regardless of whether a person committed an underlying wrong.

In this investigation, the evidence does not establish that the President was involved in an underlying crime related to Russian election interference. But the evidence does point to a range of other possible personal motives animating the President's conduct. These include concerns that continued investigation would call into question the legitimacy of his election and potential uncertainty about whether certain events—such as advance notice of WikiLeaks's release of hacked informationorthe June 9, 2016 meeting between senior campaign officials and Russians— could be seen as criminal activity by the President, his campaign, or his family.

Third, many of the President's acts directed at witnesses, including discouragement of cooperation with the government and suggestions of possible future pardons, occurred in public view. While it may be more difficult to establish that public-facing acts were motivated by a corrupt intent, the President's power to influence actions, persons, and events is enhanced by his unique ability to attract attention through use of mass communications. And no principle of law excludes public acts from the scope of obstruction statutes. If the likely effect of the acts is to intimidate witnessesoralter their testimony, the justice system's integrity is equally threatened.

Although the events we investigated involved discrete acts—e.g., the President's statement to Comey about the Flynn investigation, his termination of Comey, and his efforts to removetheSpecialCounsel—itisimportanttoviewthePresident'spatternofconductasawhole. That pattern shed slight on the nature of the President's acts and the inferences that can be drawn about his intent.

OurinvestigationfoundmultipleactsbythePresidentthatwerecapableofexerting undue influence over law enforcement investigations, including the Russian-interference and obstruction investigations. The incidents were often carried out through one-on-one meetings in which the President sought to use his official power outside of usual channels. These actions ranged from efforts to remove the Special Counsel and to reverse the effect of the Attorney General's recusal; to the attempted use of official power to limit the scope of the investigation; to direct and indirect contacts with witnesses with the potential to influence their testimony. Viewing
the acts collectively can help to illuminate their significance. For example, the President's direction to McGahn to have the Special Counsel removed was followed almost immediately by his direction to Lewandowski to tell the Attorney General to limit the scope of the Russia investigation to prospective election-interference only—a temporal connection that suggests that both acts were taken with a related purpose with respect to the investigation.

The President's efforts to influence the investigation were mostly unsuccessful, but that is largely because the persons who surrounded the President declined to carry out orders or accede to his requests. Comey did not end the investigation of Flynn, which ultimately resulted in Flynn's prosecution and conviction for lying to the FBI. McGahn did not tell the Acting Attorney General that the Special Counsel must be removed, but was instead prepared to resign over the President's order. Lewandowski and Dearborn did not deliver the President's message to Sessions that he should confine the Russia investigation to future election meddling only. And McGahn refused to recede from his recollections about events surrounding the President's direction to have the Special Counsel removed, despite the President's multiple demands that he do so. Consistent with that pattern, the evidence we obtained would not support potential obstruction charges against the President's aides and associates beyond thosealreadyfiled.

In considering the full scope of the conduct we investigated, the President's actions can be divided into two distinct phases reflecting a possible shift in the President's motives. In the first phase, before the President fired Comey, the President had been assured that the FBI had not opened an investigation of him personally. The President deemed it critically important to make public that he was not under investigation, and he included that information in his termination letter to Comey after other efforts to have that information disclosed were unsuccessful.

Soon after he fired Comey, however, the President became aware that investigators were conducting an obstruction-of-justice inquiry into his own conduct. That awareness marked a significant change in the President's conduct and the start of a second phase of action. The President launched public attacks on the investigation and individuals involved in it who could possess evidence adverse to the President, while in private, the President engaged in a series of targeted efforts to control the investigation. For instance, the President attempted to remove the Special Counsel; he sought to have Attorney General Sessions unrecuse himself and limit the investigation;hesoughttopreventpublicdisclosureofinformationabouttheJune9, 2016meeting between Russians and campaign officials; and he used public forums to attack potential witnesses who might offer adverse information and to praise witnesses who declined to cooperate with the government. Judgments about the nature of the President's motives during each phase would be informed by the totality of the evidence.

