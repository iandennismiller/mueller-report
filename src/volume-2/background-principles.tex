\section{Background Legal and Evidentiary Principles}
\markboth{Background Legal and Evidentiary Principles}{Background Legal and Evidentiary Principles}

\subsection{Legal Framework of Obstruction Of Justice}

\hyperlink{section.3.1}{The May~17, 2017 Appointment Order} and the Special Counsel regulations provide this Office with jurisdiction to investigate ``federal crimes committed in the course of, and with intent to interfere with, the Special Counsel's investigation, such as perjury, obstruction of justice, destruction of evidence, and intimidation of witnesses.''
28~C.F.R. \S~600.4(a).
Because of that description of our jurisdiction, we sought evidence for our obstruction-of-justice investigation with the elements of obstruction offenses in mind.
Our evidentiary analysis is similarly focused on the elements of such offenses, although we do not draw conclusions on the ultimate questions that govern a prosecutorial decision under the Principles of Federal Prosecution.
\textit{See} Justice Manual \S~9-27.000 \textit{et seq.}~(2018).

Here, we summarize the law interpreting the elements of potentially relevant obstruction statutes in an ordinary case.
This discussion does not address the unique constitutional issues that arise in an inquiry into official acts by the President.
Those issues are discussed in a later section of this report addressing constitutional defenses that the President's counsel have raised. \textit{See} \hyperlink{subsection.2.3.2}{Volume~II, Section~III.B}, \textit{infra}.

Three basic elements are common to most of the relevant obstruction statutes: (1)~an obstructive act; (2)~a nexus between the obstructive act and an official proceeding; and (3)~a corrupt intent.
\textit{See, e.g.}, 18~U.S.C. \S\S~1503, 1505,~1512(c)(2).
We describe those elements as they have been interpreted by the courts.
We then discuss a more specific statute aimed at witness tampering, \textit{see} 18~U.S.C. \S~1512(b), and describe the requirements for attempted offenses and endeavors to obstruct justice, \textit{see} 18~U.S.C. \S\S~1503,~1512(c)(2).

\subsubsection*{Obstructive act.}

Obstruction-of-justice law ``reaches all corrupt conduct capable of producing an effect that prevents justice from being duly administered, regardless of the means employed.''
\textit{United States~v.\ Silverman}, 745~F.2d 1386, 1393 (11th~Cir.~1984) (interpreting 18~U.S.C. \S~1503).
An ``effort to influence'' a proceeding can qualify as an endeavor to obstruct justice even if the effort was ``subtle or circuitous'' and ``however cleverly or with whatever cloaking of purpose'' it was made.
\textit{United States~v.\ Roe}, 529~F.2d 629, 632 (4th~Cir.~1975);
\textit{see also United States~v.\ Quattrone}, 441~F.3d 153, 173 (2d~Cir.~2006).
The verbs ``\thinspace`obstruct or impede' are broad'' and ``can refer to anything that blocks, makes difficult, or hinders.''
\textit{Marinello~v.\ United States}, 138~S.~Ct.\ 1101, 1106~(2018) (internal brackets and quotation marks omitted).

An improper motive can render an actor's conduct criminal even when the conduct would otherwise be lawful and within the actor's authority.
\textit{See United States~v.\ Cueto}, 151~F.3d 620, 631 (7th~Cir.~1998) (affirming obstruction conviction of a criminal defense attorney for ``litigation related conduct''); \textit{United States~v.\ Cintolo}, 818~F.2d 980, 992 (1st~Cir.~1987) (``any act by any party---whether lawful or unlawful on its face---may abridge \S~1503 if performed with a corrupt motive'').

\subsubsection*{Nexus to a pending or contemplated official proceeding.}

Obstruction-of-justice law generally requires a nexus, or connection, to an official proceeding.
In Section~1503, the nexus must be to pending ``judicial or grand jury proceedings.''
\textit{United States~v.\ Aguilar}, 515~U.S. 593, 599~(1995).
In Section~1505, the nexus can include a connection to a ``pending'' federal agency proceeding or a congressional inquiry or investigation.
Under both statutes, the government must demonstrate ``a relationship in time, causation, or logic'' between the obstructive act and the proceeding or inquiry to be obstructed.
\textit{Id.}~at~599; \textit{see also Arthur Andersen LLP~v.\ United States}, 544~U.S. 696, 707--708~(2005).
Section~1512(c) prohibits obstructive efforts aimed at official proceedings including judicial or grand jury proceedings.
18~U.S.C. \S~1515(a)(1)(A).
``For purposes of\thinspace'' Section~1512, ``an official proceeding need not be pending or about to be instituted at the time of the offense.''
18~U.S.C. \S~1512(f)(1).
Although a proceeding need not already be in progress to trigger liability under Section~1512(c), a nexus to a contemplated proceeding still must be shown.
\textit{United States~v.\ Young}, 916~F.3d 368, 386 (4th~Cir.~2019);
\textit{United States~v.\ Petruk}, 781~F.3d 438, 445 (8th~Cir.~2015);
\textit{United States~v.\ Phillips}, 583~F.3d 1261, 1264 (10th~Cir.~2009);
\textit{United States~v.\ Reich}, 479~F.3d 179, 186 (2d~Cir.~2007).
The nexus requirement narrows the scope of obstruction statutes to ensure that individuals have ``fair warning'' of what the law proscribes.
\textit{Aguilar}, 515~U.S. at~600 (internal quotation marks omitted).

The nexus showing has subjective and objective components.
As an objective matter, a defendant must act ``in a manner that is likely to obstruct justice,'' such that the statute ``excludes defendants who have an evil purpose but use means that would only unnaturally and improbably be successful.''
\textit{Aguilar}, 515~U.S. at~601--602 (emphasis added; internal quotation marks omitted).
``(T]he endeavor must have the natural and probable effect of interfering with the due administration of justice.''
\textit{Id.}~at~599 (citation and internal quotation marks omitted).
As a subjective matter, the actor must have ``contemplated a particular, foreseeable proceeding.''
\textit{Petruk}, 781~F.3d at~445--446.
A defendant need not directly impede the proceeding.
Rather, a nexus exists if ``discretionary actions of a third person would be required to obstruct the judicial proceeding if it was foreseeable to the defendant that the third party would act on the [defendant's] communication in such a way as to obstruct the judicial proceeding.''
\textit{United States~v.\ Martinez}, 862~F.3d 223, 238 (2d~Cir.~2017) (brackets, ellipses, and internal quotation marks omitted).

\subsubsection*{Corruptly.}

The word ``corruptly'' provides the intent element for obstruction of justice and means acting ``knowingly and dishonestly'' or ``with an improper motive.''
\textit{United States~v.\ Richardson}, 676~F.3d 491, 508 (5th~Cir.~2012);
\textit{United States~v.\ Gordon}, 710~F.3d 1124, 1151 (10th~Cir.~2013) (to act corruptly means to ``act[] with an improper purpose and to engage in conduct knowingly and dishonestly with the specific intent to subvert, impede or obstruct'' the relevant proceeding) (some quotation marks omitted);
\textit{see} 18~U.S.C. \S~1515(b) (``As used in section 1505, the term `corruptly' means acting with an improper purpose, personally or by influencing another.'');
\textit{see also Arthur Andersen}, 544~U.S. at~705--706 (interpreting ``corruptly'' to mean ``wrongful, immoral, depraved, or evil'' and holding that acting ``knowingly \dots\ corruptly'' in 18~U.S.C. \S~1512(b) requires ``consciousness of wrongdoing'').
The requisite showing is made when a person acted with an intent to obtain an ``improper advantage for [him]self or someone else, inconsistent with official duty and the rights of others.''
\textsc{Ballentine's Law Dictionary} 276 (3d ed.~1969);
\textit{see United States~v.\ Pasha}, 797~F.3d 1122, 1132 (D.C.~Cir.~2015);
\textit{Aguilar}, 515~U.S. at~616 (Scalia,~J., concurring in part and dissenting in part) (characterizing this definition as the ``longstanding and well-accepted meaning'' of ``corruptly'').

\subsubsection*{Witness tampering.}

A more specific provision in Section~1512 prohibits tampering with a witness.
\textit{See} 18~U.S.C. \S~1512(b)(1), (3) (making it a crime to ``knowingly use[] intimidation \dots\ or corruptly persuade[] another person,'' or ``engage[] in misleading conduct towards another person,'' with the intent to ``influence, delay, or prevent the testimony of any person in an official proceeding'' or to ``hinder, delay, or prevent the communication to a law enforcement officer \dots\ of information relating to the commission or possible commission of a Federal offense'').
To establish corrupt persuasion, it is sufficient that the defendant asked a potential witness to lie to investigators in contemplation of a likely federal investigation into his conduct.
\textit{United States~v.\ Edlind}, 887~F.3d 166, 174 (4th~Cir.~2018);
\textit{United States~v.\ Sparks}, 791~F.3d 1188, 1191--1192 (10th~Cir.~2015);
\textit{United States~v.\ Byrne}, 435~F.3d 16, 23--26 (1st~Cir.~2006);
\textit{United States~v.\ LaShay}, 417~F.3d 715, 718--719 (7th~Cir.~2005);
\textit{United States~v.\ Burns}, 298~F.3d 523, 539--540 (6th~Cir.~2002);
\textit{United States~v.\ Pennington}, 168~F.3d 1060, 1066 (8th~Cir.~1999).
The ``persuasion'' need not be coercive, intimidating, or explicit;
it is sufficient to ``urge,'' ``induce,'' ``ask[],'' ``argu[e],'' ``giv[e] reasons,''
\textit{Sparks}, 791~F.3d at~1192, or ``coach[] or remind[] witnesses by planting misleading facts,''
\textit{Edlind}, 887~F.3d at~174.
Corrupt persuasion is shown ``where a defendant tells a potential witness a false story as if the story were true, intending that the witness believe the story and testify to it.''
\textit{United States~v.\ Rodolitz}, 786~F.2d 77, 82 (2d~Cir.~1986);
\textit{see United States~v.\ Gabriel}, 125~F.3d 89, 102 (2d~Cir.~1997).
It also covers urging a witness to recall a fact that the witness did not know, even if the fact was actually true.
\textit{See LaShay}, 417~F.3d at~719.
Corrupt persuasion also can be shown in certain circumstances when a person, with an improper motive, urges a witness not to cooperate with law enforcement.
\textit{See United States~v.\ Shotts}, 145~F.3d 1289, 1301 (11th Cr.~1998) (telling Secretary ``not to [say] anything [to the FBI] and [she] would not be bothered'').

When the charge is acting with the intent to hinder, delay, or prevent the communication of information to law enforcement under Section 1512(b)(3), the ``nexus'' to a proceeding inquiry articulated in \textit{Aguilar}---that an individual have ``knowledge that his actions are likely to affect the judicial proceeding,'' 515~U.S. at~599---does not apply because the obstructive act is aimed at the communication of information to investigators, not at impeding an official proceeding.

Acting ``knowingly \dots\ corruptly'' requires proof that the individual was ``conscious of wrongdoing.''
\textit{Arthur Andersen}, 544~U.S. at~705--706 (declining to explore ``[t]he outer limits of this element'' but indicating that an instruction was infirm where it permitted conviction even if the defendant ``honestly and sincerely believed that [the] conduct was lawful'').
It is an affirmative defense that ``the conduct consisted solely of lawful conduct and that the defendant's sole intention was to encourage, induce, or cause the other person to testify truthfully.''
18~U.S.C. \S~1512(e).

\subsubsection*{Attempts and endeavors.}

Section 1512(c)(2) covers both substantive obstruction offenses and attempts to obstruct justice. Under general principles of attempt law, a person is guilty of an attempt when he has the intent to commit a substantive offense and takes an overt act that constitutes a substantial step towards that goal.
\textit{See United States~v.\ Resendiz-Ponce}, 549~U.S. 102, 106--107~(2007).
``[T]he act [must be] substantial, in that it was strongly corroborative of the defendant's criminal purpose.''
\textit{United States~v.\ Pratt}, 351~F.3d 131, 135 (4th~Cir.~2003).
While ``mere abstract talk'' does not suffice, any ``concrete and specific'' acts that corroborate the defendant's intent can constitute a ``substantial step.''
\textit{United States~v.\ Irving}, 665~F.3d 1184, 1198--1205 (10th~Cir.~2011).
Thus, ``soliciting an innocent agent to engage in conduct constituting an element of the crime'' may qualify as a substantial step.
Model Penal Code \S~5.01(2)(g);
\textit{see United States~v.\ Lucas}, 499~F.3d 769, 781 (8th~Cir.~2007).

The omnibus clause of 18~U.S.C. \S~1503 prohibits an ``endeavor'' to obstruct justice, which sweeps more broadly than Section~1512's attempt provision.
\textit{See United States~v.\ Sampson}, 898~F.3d 287, 302 (2d~Cir.~2018);
\textit{United States~v.\ Leisure}, 844~F.2d 1347, 1366--1367 (8th~Cir.~1988) (collecting cases).
``It is well established that a[n] [obstruction-of-justice] offense is complete when one corruptly endeavors to obstruct or impede the due administration of justice;
the prosecution need not prove that the due administration of justice was actually obstructed or impeded.''
\textit{United States~v.\ Davis}, 854~F.3d 1276, 1292 (11th~Cir.~2017) (internal quotation marks omitted).

\subsection{Investigative and Evidentiary Considerations}

After the appointment of the Special Counsel, this Office obtained evidence about the following events relating to potential issues of obstruction of justice involving the President:

\renewcommand{\labelenumi}{(\alph{enumi})}
\begin{enumerate}
    \item The President's January~27, 2017 dinner with former FBI Director James Comey in which the President reportedly asked for Comey's loyalty, one day after the White House had been briefed by the Department of Justice on contacts between former National Security Advisor Michael Flynn and the Russian Ambassador;

    \item The President's February~14, 2017 meeting with Comey in which the President reportedly asked Comey not to pursue an investigation of Flynn;

    \item The President's private requests to Comey to make public the fact that the President was not the subject of an FBI investigation and to lift what the President regarded as a cloud;

    \item The President's outreach to the Director of National Intelligence and the Directors of the National Security Agency and the Central Intelligence Agency about the FBI's Russia investigation;

    \item The President's stated rationales for terminating Comey on May~9, 2017, including statements that could reasonably be understood as acknowledging that the FBI's Russia investigation was a factor in Comey's termination; and

    \item The President's reported involvement in issuing a statement about the June~9, 2016 Trump Tower meeting between Russians and senior Trump Campaign officials that said the meeting was about adoption and omitted that the Russians had offered to provide the Trump Campaign with derogatory information about Hillary Clinton.
\end{enumerate}

Taking into account that information and our analysis of applicable statutory and constitutional principles (discussed below in \hyperlink{section.2.3}{Volume~II, Section~III}, \textit{infra}), we determined that there was a sufficient factual and legal basis to further investigate potential obstruction-of-justice issues involving the President.

Many of the core issues in an obstruction-of-justice investigation turn on an individual's actions and intent.
We therefore requested that the White House provide us with documentary evidence in its possession on the relevant events.
We also sought and obtained the White House's concurrence in our conducting interviews of White House personnel who had relevant information.
And we interviewed other witnesses who had pertinent knowledge, obtained documents on a voluntary basis when possible, and used legal process where appropriate.
These investigative steps allowed us to gather a substantial amount of evidence.

We also sought a voluntary interview with the President.
After more than a year of discussion, the President declined to be interviewed.
\blackout{Grand Jury}
During the course of our discussions, the President did agree to answer written questions on certain Russia-related topics, and he provided us with answers.
He did not similarly agree to provide written answers to questions on obstruction topics or questions on events during the transition.
Ultimately, while we believed that we had the authority and legal justification to issue a grand jury subpoena to obtain the President's testimony, we chose not to do so.
We made that decision in view of the substantial delay that such an investigative step would likely produce at a late stage in our investigation.
We also assessed that based on the significant body of evidence we had already obtained of the President's actions and his public and private statements describing or explaining those actions, we had sufficient evidence to understand relevant events and to make certain assessments without the President's testimony.
The Office's decision-making process on this issue is described in more detail in \hyperlink{section.3.3}{Appendix~C}, \textit{infra}, in a note that precedes the President's written responses.

In assessing the evidence we obtained, we relied on common principles that apply in any investigation.
The issue of criminal intent is often inferred from circumstantial evidence.
\textit{See, e.g., United States~v.\ Croteau}, 819~F.3d 1293, 1305 (11th~Cir.~2016) (``[G]uilty knowledge can rarely be established by direct evidence\dots.
Therefore, mens rea elements such as knowledge or intent may be proved by circumstantial evidence.'') (internal quotation marks omitted);
\textit{United States~v.\ Robinson}, 702~F.3d 22, 36 (2d~Cir.~2012) (``The government's case rested on circumstantial evidence, but the mens rea elements of knowledge and intent can often be proved through circumstantial evidence and the reasonable inferences drawn therefrom.'') (internal quotation marks omitted).
The principle that intent can be inferred from circumstantial evidence is a necessity in criminal cases, given the right of a subject to assert his privilege against compelled self-incrimination under the Fifth Amendment and therefore decline to testify.
Accordingly, determinations on intent are frequently reached without the opportunity to interview an investigatory subject.

Obstruction-of-justice cases are consistent with this rule.
\textit{See, e.g., Edlind}, 887~F.3d at~174, 176 (relying on ``significant circumstantial evidence that [the defendant] was conscious of her wrongdoing'' in an obstruction case; ``[b]ecause evidence of intent will almost always be circumstantial, a defendant may be found culpable where the reasonable and foreseeable consequences of her acts are the obstruction of justice'') (internal quotation marks, ellipses, and punctuation omitted);
\textit{Quattrone}, 441~F.3d at~173--174.
Circumstantial evidence that illuminates intent may include a pattern of potentially obstructive acts. Fed.~R. Evid.~404(b) (``Evidence of a crime, wrong, or other act \dots\ may be admissible \dots\ [to] prov[e] motive, opportunity, intent, preparation, plan, knowledge, identity, absence of mistake, or lack of accident.'');
\textit{see, e.g., United States~v.\ Frankhauser}, 80~F.3d 641, 648--650 (1st~Cir.~1996);
\textit{United States~v.\ Arnold}, 773~F.2d 823, 832--834 (7th~Cir.~1985);
\textit{Cintolo}, 818~F.2d at~1000.

Credibility judgments may also be made based on objective facts and circumstantial evidence.
Standard jury instructions highlight a variety of factors that are often relevant in assessing credibility.
These include whether a witness had a reason not to tell the truth; whether the witness had a good memory;
whether the witness had the opportunity to observe the events about which he testified;
whether the witness's testimony was corroborated by other witnesses;
and whether anything the witness said or wrote previously contradicts his testimony.
\textit{See, e.g., First Circuit Pattern Jury Instructions} \S~1.06~(2018);
\textit{Fifth Circuit Pattern Jury Instructions (Criminal Cases)} \S~1.08~(2012);
\textit{Seventh Circuit Pattern Jury Instruction} \S~3.01~(2012).

In addition to those general factors, we took into account more specific factors in assessing the credibility of conflicting accounts of the facts.
For example, contemporaneous written notes can provide strong corroborating evidence.
\textit{See United States~v.\ Nobles}, 422~U.S. 225, 232~(1975) (the fact that a ``statement appeared in the contemporaneously recorded report \dots\ would tend strongly to corroborate the investigator's version of the interview'').
Similarly, a witness's recitation of his account before he had any motive to fabricate also supports the witness's credibility.
\textit{See Tome~v.\ United States}, 513~U.S. 150, 158~(1995) (``A consistent statement that predates the motive is a square rebuttal of the charge that the testimony was contrived as a consequence of that motive.'').
Finally, a witness's false description of an encounter can imply consciousness of wrongdoing.
\textit{See Al-Adahi~v.\ Obama}, 613~F.3d 1102, 1107 (D.C.~Cir.~2010) (noting the ``well-settled principle that false exculpatory statements are evidence---often strong evidence---of guilt'').
We applied those settled legal principles in evaluating the factual results of our investigation.
